\documentclass[a4paper]{article}
\usepackage{setspace}
\usepackage[utf8]{inputenc}
\usepackage[russian]{babel}
\usepackage{graphicx}
\usepackage[12pt]{extsizes}
\usepackage{mathtools}
\usepackage{graphicx}
\usepackage{fancyhdr}
\usepackage{amssymb}
\usepackage{amsmath, amsfonts, amssymb, amsthm, mathtools}
\usepackage{tikz}

\usetikzlibrary{positioning}
\setstretch{1.3}

\newcommand{\mat}[1]{\begin{pmatrix} #1 \end{pmatrix}}
\newcommand{\vmat}[1]{\begin{vmatrix} #1 \end{vmatrix}}
\renewcommand{\f}[2]{\frac{#1}{#2}}
\newcommand{\dspace}{\space\space}
\newcommand{\s}[2]{\sum\limits_{#1}^{#2}}
\newcommand{\mul}[2]{\prod_{#1}^{#2}}
\newcommand{\sq}[1]{\left[ {#1} \right]}
\newcommand{\gath}[1]{\left[ \begin{array}{@{}l@{}} #1 \end{array} \right.}
\newcommand{\case}[1]{\begin{cases} #1 \end{cases}}
\newcommand{\ts}{\text{\space}}
\newcommand{\lm}[1]{\underset{#1}{\lim}}
\newcommand{\suplm}[1]{\underset{#1}{\overline{\lim}}}
\newcommand{\inflm}[1]{\underset{#1}{\underline{\lim}}}
\newcommand{\Ker}[1]{\operatorname{Ker}}

\renewcommand{\phi}{\varphi}
\newcommand{\lr}{\Leftrightarrow}
\renewcommand{\l}{\left(}
\renewcommand{\r}{\right)}
\newcommand{\rr}{\rightarrow}
\renewcommand{\geq}{\geqslant}
\renewcommand{\leq}{\leqslant}
\newcommand{\RR}{\mathbb{R}}
\newcommand{\CC}{\mathbb{C}}
\newcommand{\QQ}{\mathbb{Q}}
\newcommand{\ZZ}{\mathbb{Z}}
\newcommand{\VV}{\mathbb{V}}
\newcommand{\NN}{\mathbb{N}}
\newcommand{\OO}{\underline{O}}
\newcommand{\oo}{\overline{o}}
\renewcommand{\Ker}{\operatorname{Ker}}
\renewcommand{\Im}{\operatorname{Im}}
\newcommand{\vol}{\text{vol}}
\newcommand{\Vol}{\text{Vol}}

\DeclarePairedDelimiter\abs{\lvert}{\rvert} %
\makeatletter                               % \abs{}
\let\oldabs\abs                             %
\def\abs{\@ifstar{\oldabs}{\oldabs*}}       %

\begin{document}

\section*{Домашнее задание на 17.06 (Алгебра)}
 {\large Емельянов Владимир, ПМИ гр №247}\\\\
\begin{enumerate}
  \item[\textbf{№1}]
  По условию
  \[
    F_9 \; = \; \ZZ_3[x]\big/\bigl(x^2 + 2x + 2\bigr),
  \]
  Обозначим класс многочлена \(a + b x\) через пару \((a,b)\), где \(a,b\in\{0,1,2\}\).

  
  Так как в \(\ZZ_3\) выполнено \(-2 \equiv 1\pmod{3}\), 
  из уравнения \(x^2 + 2x + 2 = 0\) следует
  \[
    x^2 \;\equiv\; -2x - 2 \;\equiv\; x + 1 \quad(\bmod\,3)
  \]
  Порождающие элементы имеют порядок 8. Проверим, что для каждого элемента $g$:
  $$
  g^1 \neq 1, \quad g^2 \neq 1, \quad g^4 \neq 1.
  $$
  Если это верно, то $g^8 = 1$, и $g$ — порождающий.
    
  \begin{enumerate}
    \item Элемент $x$:
      $$
      \begin{aligned}
      x^1 &= x \neq 1, \\
      x^2 &= x + 1 \neq 1, \\
      x^4 &= (x^2)^2 = (x + 1)^2 = x^2 + 2x + 1 \\ & = (x + 1) + 2x + 1 = 3x + 2 =
       2 \neq 1
      \end{aligned}
      $$
      Значит, порядок $x$ равен 8.

    \item Элемент $2x$:
      $$
      \begin{aligned}
      (2x)^1 &= 2x \neq 1, \\
      (2x)^2 &= 4x^2 = 4(x + 1) = 4x + 4 = x + 1 \neq 1, \\
      (2x)^4 &= (x + 1)^2 = 2 \neq 1
      \end{aligned}
      $$
      Порядок $2x$ равен 8.

    \item Элемент $x + 2$:
      $$
      \begin{aligned}
      (x + 2)^1 &= x + 2 \neq 1, \\
      (x + 2)^2 &= x^2 + 4x + 4 = (x + 1) + x + 1 = 2x + 2 \neq 1, \\
      (x + 2)^4 &= (2x + 2)^2 = 4x^2 + 8x + 4 = 2 \neq 1
      \end{aligned}
      $$
      Порядок $x + 2$ равен 8.

    \item Элемент $2x + 1$:
      $$
      \begin{aligned}
      (2x + 1)^1 &= 2x + 1 \neq 1, \\
      (2x + 1)^2 &= 4x^2 + 4x + 1 = (x + 1) + x + 1 = 2x + 2 \neq 1, \\
      (2x + 1)^4 &= (2x + 2)^2 = 2 \neq 1
      \end{aligned}
      $$
      Порядок $2x + 1$ равен 8.
  \end{enumerate}
  Остальные элементы:
  $1$ (порядок 1), 
  $2$ (порядок 2), 
  $x + 1$ и $2x + 2$ имеют порядок 4, так как $(x + 1)^4 = 2^2 = 1$ и $(2x + 2)^4 = 1$.

  \textbf{Ответ: } $x, \quad 2x, \quad x + 2, \quad 2x + 1$\\

  \item[\textbf{№2}]Многочлен \(x^2 + 3\) над \(\mathbb{Z}_5\):
  \begin{itemize}
  \item \(0^2 + 3 = 3 \neq 0\)
  \item \(1^2 + 3 = 4 \neq 0\)
  \item \(2^2 + 3 = 2 \neq 0\)
  \item \(3^2 + 3 = 2 \neq 0\)
  \item \(4^2 + 3 = 4 \neq 0\)
  \end{itemize}

  Нет корней, следовательно, \(x^2 + 3\) неприводим.

  Многочлен \(y^2 + y + 2\) над \(\mathbb{Z}_5\):

  \begin{itemize}
    \item \(0^2 + 0 + 2 = 2 \neq 0\)
    \item \(1^2 + 1 + 2 = 4 \neq 0\)
    \item \(2^2 + 2 + 2 = 3 \neq 0\)
    \item \(3^2 + 3 + 2 = 4 \neq 0\)
    \item \(4^2 + 4 + 2 = 2 \neq 0\)
  \end{itemize}

  Нет корней, следовательно, \(y^2 + y + 2\) неприводим.

  Обозначим:
  \begin{itemize}
  \item \(\alpha\) — корень \(x^2 + 3 = 0\) в \(\mathbb{Z}_5[x]/(x^2 + 3)\)
  \item \(\beta\) — корень \(y^2 + y + 2 = 0\) в \(\mathbb{Z}_5[y]/(y^2 + y + 2)\)
  \end{itemize}

  Найдём подстановку \(\beta = a\alpha + b\), удовлетворяющую уравнению
   \(\beta^2 + \beta + 2 = 0\). 
   
   После решения системы уравнений получаем два варианта:
  \begin{itemize}
  \item \(\beta = \alpha + 2\)
  \item \(\beta = 4\alpha + 2\)
  \end{itemize}

  Выберем \(\beta = \alpha + 2\).
  Тогда изоморфизм \(\phi: \mathbb{Z}_5[x]/(x^2 + 3) \to \mathbb{Z}_5[y]/(y^2 + y + 2)\)
  задаётся как:
  \[
  \phi(a + b\alpha) = a + b\beta = a + b(\alpha + 2)
  \]
  Проверим
  \begin{itemize}
    \item \(\beta^2 = (\alpha + 2)^2 = \alpha^2 + 4\alpha + 4 = 2 + 4\alpha + 4 = 4\alpha + 1\),
    \item \(\beta^2 + \beta + 2 = (4\alpha + 1) + (\alpha + 2) + 2 = 5\alpha + 5 \equiv 0 \mod 5\).
  \end{itemize}
  Изоморфизм сохраняет операции сложения и умножения, так как \(\beta\) удовлетворяет требуемому уравнению.

  \textbf{Ответ: } $\phi(a + b\alpha) = a + b(\alpha + 2)$\\

  \item[\textbf{№3}]Поле \(\mathbb{F}_{262144}\) имеет порядок \(2^{18}\). 
  Подполя этого поля имеют порядки \(2^d\), где \(d\) — делитель 18.
   Делители 18: \(1, 2, 3, 6, 9, 18\). 
   
  Соответствующие подполя:  
  \[
  \mathbb{F}_2, \quad \mathbb{F}_4, \quad \mathbb{F}_8, \quad \mathbb{F}_{64}, 
  \quad \mathbb{F}_{512}, \quad \mathbb{F}_{262144}
  \]
  Проверим наличие корней в подполях
  \begin{enumerate}
  \item Подполе \(\mathbb{F}_2\)
 
     Многочлен \(x^3 + x^2 + 1\) в \(\mathbb{F}_2[x]\):
    \begin{itemize}
      \item \(f(0) = 0 + 0 + 1 = 1 \neq 0\)
      \item \(f(1) = 1 + 1 + 1 = 3 \equiv 1 \neq 0\)
    \end{itemize}
    \textbf{Корней нет}

  \item Подполе \(\mathbb{F}_4\)
    
  Представим \(\mathbb{F}_4\) как \(\mathbb{F}_2[\alpha]/(\alpha^2 + \alpha + 1)\).
   Элементы: \(0, 1, \alpha, \alpha + 1\).
    \begin{itemize}
      \item \(f(0) = 1 \neq 0\)
      \item \(f(1) = 1 \neq 0\)
      \item \(f(\alpha) = \alpha^3 + \alpha^2 + 1 = (\alpha + 1) + 
      \alpha + 1 = 1 \neq 0\) (используя \(\alpha^2 = \alpha + 1\))
      \item \(f(\alpha + 1) = (\alpha + 1)^3 + (\alpha + 1)^2 + 1 = \alpha + 1 \neq 0\)
    \end{itemize}
    \textbf{Корней нет}

  \item Подполе \(\mathbb{F}_8\)

  Многочлен \(x^3 + x^2 + 1\) неприводим над \(\mathbb{F}_2\)
   (нет корней в \(\mathbb{F}_2\)). Следовательно, \(\mathbb{F}_8 \cong
    \mathbb{F}_2[x]/(x^3 + x^2 + 1)\), и корень многочлена существует в \(\mathbb{F}_8\)
    
  \textbf{Корень есть}

  \item Большие подполя
  
  Подполя \(\mathbb{F}_{64}\), \(\mathbb{F}_{512}\), \(\mathbb{F}_{262144}\) 
  содержат \(\mathbb{F}_8\) (поскольку их порядки кратны \(8 = 2^3\)). 
  Корень из \(\mathbb{F}_8\) автоматически принадлежит этим подполям.
  
  \textbf{Корень существует}
  \end{enumerate}

  \textbf{Ответ: } $\mathbb{F}_8, \mathbb{F}_{64},
  \mathbb{F}_{512}, \mathbb{F}_{262144}$\\

  \item[\textbf{№4}]
  Пусть $ \beta \in \mathbb{F}_q $ — корень многочлена
  $ f(x) = x^p - x - \alpha $.
  
  Тогда:
  $$
  \beta^p - \beta = \alpha
  $$

  В поле характеристики $ p $ выполняется тождество $(a + b)^p = a^p + b^p$.
   Для любого $ c \in \mathbb{F}_p $:
  $$
  (\beta + c)^p - (\beta + c) = \beta^p + c^p - \beta - c
  $$
  Поскольку $ c \in \mathbb{F}_p $, по малой теореме Ферма $ c^p = c $. Тогда:
  $$
  (\beta + c)^p - (\beta + c) = \beta^p - \beta = \alpha
  $$
  Это означает, что $ \beta + c $ также является корнем $ f(x) $

  
  Так как $ \beta \in \mathbb{F}_q $ и $ c \in \mathbb{F}_p \subseteq 
  \mathbb{F}_q $, все элементы вида $ \beta + c $ принадлежат 
  $ \mathbb{F}_q $. 
  
  Множество 
  $$\{\beta + c \mid c \in \mathbb{F}_p\}$$ 
  содержит ровно $ p $ различных элементов 
  (поскольку $ \beta + c_1 = \beta + c_2 \Rightarrow c_1 = c_2 $).

  Таким образом, многочлен $ f(x) $ имеет $ p $ корней в $ \mathbb{F}_q $,
   и его можно разложить на линейные множители:
  $$
  f(x) = \prod_{c \in \mathbb{F}_p} (x - (\beta + c))
  $$
\end{enumerate}
\end{document}