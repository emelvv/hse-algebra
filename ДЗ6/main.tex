\documentclass[a4paper]{article}
\usepackage{setspace}
\usepackage[utf8]{inputenc}
\usepackage[russian]{babel}
\usepackage[12pt]{extsizes}
\usepackage{mathtools}
\usepackage{graphicx}
\usepackage{fancyhdr}
\usepackage{amssymb}
\usepackage{amsmath, amsfonts, amssymb, amsthm, mathtools}
\usepackage{tikz}

\usetikzlibrary{positioning}
\setstretch{1.3}

\newcommand{\mat}[1]{\begin{pmatrix} #1 \end{pmatrix}}
\newcommand{\vmat}[1]{\begin{vmatrix} #1 \end{vmatrix}}
\renewcommand{\f}[2]{\frac{#1}{#2}}
\newcommand{\dspace}{\space\space}
\newcommand{\s}[2]{\sum\limits_{#1}^{#2}}
\newcommand{\mul}[2]{\prod_{#1}^{#2}}
\newcommand{\sq}[1]{\left[ {#1} \right]}
\newcommand{\gath}[1]{\left[ \begin{array}{@{}l@{}} #1 \end{array} \right.}
\newcommand{\case}[1]{\begin{cases} #1 \end{cases}}
\newcommand{\ts}{\text{\space}}
\newcommand{\lm}[1]{\underset{#1}{\lim}}
\newcommand{\suplm}[1]{\underset{#1}{\overline{\lim}}}
\newcommand{\inflm}[1]{\underset{#1}{\underline{\lim}}}
\newcommand{\Ker}[1]{\operatorname{Ker}}

\renewcommand{\phi}{\varphi}
\newcommand{\lr}{\Leftrightarrow}
\renewcommand{\l}{\left(}
\renewcommand{\r}{\right)}
\newcommand{\rr}{\rightarrow}
\renewcommand{\geq}{\geqslant}
\renewcommand{\leq}{\leqslant}
\newcommand{\RR}{\mathbb{R}}
\newcommand{\CC}{\mathbb{C}}
\newcommand{\QQ}{\mathbb{Q}}
\newcommand{\ZZ}{\mathbb{Z}}
\newcommand{\VV}{\mathbb{V}}
\newcommand{\NN}{\mathbb{N}}
\newcommand{\OO}{\underline{O}}
\newcommand{\oo}{\overline{o}}
\renewcommand{\Ker}{\operatorname{Ker}}
\renewcommand{\Im}{\operatorname{Im}}
\newcommand{\vol}{\text{vol}}
\newcommand{\Vol}{\text{Vol}}

\DeclarePairedDelimiter\abs{\lvert}{\rvert} %
\makeatletter                               % \abs{}
\let\oldabs\abs                             %
\def\abs{\@ifstar{\oldabs}{\oldabs*}}       %

\begin{document}

\section*{Домашнее задание на 22.05 (Алгебра)}
{\large Емельянов Владимир, ПМИ гр №247}\\\\
\begin{enumerate}
  \item[\textbf{№1}]Утверждение, что $F$ - поле эквивалетно тому, что многочлен
  $$f(z) = z^3 -z^2 + 1$$
  неприводим. При этом:
  $$f(z) = z^3 -z^2 + 1\text{ неприводим } \quad \lr \quad \text{ у $f$ нет корней в $\QQ$}$$
  Но, так как $$f(\pm 1) \neq 0 \implies \text{ у $f$ --- нет корней в $Q$} \implies f \text{ --- неприводим}$$
  Следовательно, $F$ --- поле

  Теперь нам известно, что:
  $$\alpha = z + (f(z))$$
  Значит, мы можем представить:
  $$\f{2\alpha^2 - 8\alpha + 9}{\alpha^2 -3\alpha + 1} = a\alpha^2 + b\alpha + c$$
  Осталось найти коэфиценты $a, b, c$:
  $$2\alpha^2 - 8\alpha + 9= (a\alpha^2 + b\alpha + c)(\alpha^2 -3\alpha + 1)=$$
  $$
  = a\alpha^4 -3a\alpha^3 + a\alpha^2
  + b\alpha^3 -3b\alpha^2 + b\alpha
  + c\alpha^2 -3c\alpha + c
  $$
  но, так как $\alpha^3 -\alpha^2 + 1 = 0 \implies \alpha^3 = \alpha^2 - 1$, то
  $$a\alpha^4 -3a\alpha^3 + a\alpha^2
  + b\alpha^3 -3b\alpha^2 + b\alpha
  + c\alpha^2 -3c\alpha + c =$$
  $$=a\alpha(\alpha^2 - 1) -3a\alpha^3 + a\alpha^2
  + b\alpha^3 -3b\alpha^2 + b\alpha
  + c\alpha^2 -3c\alpha + c $$
  $$= (-a-2b+c)\,\alpha^2\;+\;(-a+b-3c)\,\alpha\;+\;(2a - b + c)$$
  Составим систему:
  $$
  \begin{cases}
  -a-2b+c=2,\\
  -a+b-3c=-8,\\
  2a - b + c=9,
  \end{cases}
  \implies a=3,\quad b=-2,\quad c=1$$
  Следовательно:
  $$\f{2\alpha^2 - 8\alpha + 9}{\alpha^2 -3\alpha + 1} = 3\alpha^2 -2\alpha + 1$$
  \textbf{Ответ: } $3\alpha^2 -2\alpha + 1$

  \item[\textbf{№2}]
  По условию:
  $$g = x_2^4x_3^5 + 2x_1x_2^4x_3 + x_1^2x_2^2, \quad f = x_2^4x_3 - 2x_1x_2x_3^2 + x_1x_2^2$$
  Так как
  $$LT(g) = x_1^2 x_2^2$$
  то 
  $$g \to g-x_1f = x_2^4x_3^5 + 2x_1x_2^4x_3 + x_1^2x_2^2 - (x_1x_2^4x_3 - 2x_1^2x_2x_3^2 + x_1^2x_2^2) = $$
  $$=x_2^4x_3^5 + x_1x_2^4x_3 + 2x_1^2x_2x_3^2 = g_1 $$
  Так как 
  $$LT(g_1) = 2x_1^2x_2x_3^2$$
  то
  $$g_1 \to g_1 + x_1f = x_2^4x_3^5 + x_1x_2^4x_3 + 2x_1^2x_2x_3^2  + x_1(x_2^4x_3 - 2x_1x_2x_3^2 + x_1x_2^2) = $$
  $$=x_2^4x_3^5 + x_1x_2^4x_3 + 2x_1^2x_2x_3^2  + x_1x_2^4x_3 - 2x_1^2x_2x_3^2 + x_1^2x_2^2 = $$
  $$ = x_2^4x_3^5 + 2x_1x_2^4x_3 + x_1^2x_2^2 = g$$
  Следовательно, мы вернулись к $g$, значит:
  $$r = g_1 = x_2^4x_3^5 + x_1x_2^4x_3 + 2x_1^2x_2x_3^2$$
  \textbf{Ответ: } $x_2^4x_3^5 + x_1x_2^4x_3 + 2x_1^2x_2x_3^2$

\end{enumerate}
\end{document}