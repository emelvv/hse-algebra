\documentclass[a4paper]{article}
\usepackage{setspace}
\usepackage[utf8]{inputenc}
\usepackage[russian]{babel}
\usepackage[12pt]{extsizes}
\usepackage{mathtools}
\usepackage{graphicx}
\usepackage{fancyhdr}
\usepackage{amssymb}
\usepackage{amsmath, amsfonts, amssymb, amsthm, mathtools}
\usepackage{tikz}

\usetikzlibrary{positioning}
\setstretch{1.3}

\newcommand{\mat}[1]{\begin{pmatrix} #1 \end{pmatrix}}
\newcommand{\vmat}[1]{\begin{vmatrix} #1 \end{vmatrix}}
\renewcommand{\f}[2]{\frac{#1}{#2}}
\newcommand{\dspace}{\space\space}
\newcommand{\s}[2]{\sum\limits_{#1}^{#2}}
\newcommand{\mul}[2]{\prod_{#1}^{#2}}
\newcommand{\sq}[1]{\left[ {#1} \right]}
\newcommand{\gath}[1]{\left[ \begin{array}{@{}l@{}} #1 \end{array} \right.}
\newcommand{\case}[1]{\begin{cases} #1 \end{cases}}
\newcommand{\ts}{\text{\space}}
\newcommand{\lm}[1]{\underset{#1}{\lim}}
\newcommand{\suplm}[1]{\underset{#1}{\overline{\lim}}}
\newcommand{\inflm}[1]{\underset{#1}{\underline{\lim}}}
\newcommand{\Ker}[1]{\operatorname{Ker}}

\renewcommand{\phi}{\varphi}
\newcommand{\lr}{\Leftrightarrow}
\renewcommand{\l}{\left(}
\renewcommand{\r}{\right)}
\newcommand{\rr}{\rightarrow}
\renewcommand{\geq}{\geqslant}
\renewcommand{\leq}{\leqslant}
\newcommand{\RR}{\mathbb{R}}
\newcommand{\CC}{\mathbb{C}}
\newcommand{\QQ}{\mathbb{Q}}
\newcommand{\ZZ}{\mathbb{Z}}
\newcommand{\VV}{\mathbb{V}}
\newcommand{\NN}{\mathbb{N}}
\newcommand{\OO}{\underline{O}}
\newcommand{\oo}{\overline{o}}
\renewcommand{\Ker}{\operatorname{Ker}}
\renewcommand{\Im}{\operatorname{Im}}
\newcommand{\vol}{\text{vol}}
\newcommand{\Vol}{\text{Vol}}

\DeclarePairedDelimiter\abs{\lvert}{\rvert} %
\makeatletter                               % \abs{}
\let\oldabs\abs                             %
\def\abs{\@ifstar{\oldabs}{\oldabs*}}       %

\begin{document}

\section*{Домашнее задание на 22.05 (Алгебра)}
{\large Емельянов Владимир, ПМИ гр №247}\\\\
\begin{enumerate}
  \item[\textbf{№1}]Утверждение, что $F$ - поле эквивалетно тому, что многочлен
  $$f(z) = z^3 -z^2 + 1$$
  неприводим. При этом:
  $$f(z) = z^3 -z^2 + 1\text{ неприводим } \quad \lr \quad \text{ у $f$ нет корней в $\QQ$}$$
  Но, так как $$f(\pm 1) \neq 0 \implies \text{ у $f$ --- нет корней в $Q$} \implies f \text{ --- неприводим}$$
  Следовательно, $F$ --- поле

  Теперь нам известно, что:
  $$\alpha = z + (f(z))$$
  Значит, мы можем представить:
  $$\f{2\alpha^2 - 8\alpha + 9}{\alpha^2 -3\alpha + 1} = a\alpha^2 + b\alpha + c$$
  Осталось найти коэфиценты $a, b, c$:
  $$2\alpha^2 - 8\alpha + 9= (a\alpha^2 + b\alpha + c)(\alpha^2 -3\alpha + 1)=$$
  $$
  = a\alpha^4 -3a\alpha^3 + a\alpha^2
  + b\alpha^3 -3b\alpha^2 + b\alpha
  + c\alpha^2 -3c\alpha + c
  $$
  но, так как $\alpha^3 -\alpha^2 + 1 = 0 \implies \alpha^3 = \alpha^2 - 1$, то
  $$a\alpha^4 -3a\alpha^3 + a\alpha^2
  + b\alpha^3 -3b\alpha^2 + b\alpha
  + c\alpha^2 -3c\alpha + c =$$
  $$=a\alpha(\alpha^2 - 1) -3a\alpha^3 + a\alpha^2
  + b\alpha^3 -3b\alpha^2 + b\alpha
  + c\alpha^2 -3c\alpha + c $$
  $$= (-a-2b+c)\,\alpha^2\;+\;(-a+b-3c)\,\alpha\;+\;(2a - b + c)$$
  Составим систему:
  $$
  \begin{cases}
  -a-2b+c=2,\\
  -a+b-3c=-8,\\
  2a - b + c=9,
  \end{cases}
  \implies a=3,\quad b=-2,\quad c=1$$
  Следовательно:
  $$\f{2\alpha^2 - 8\alpha + 9}{\alpha^2 -3\alpha + 1} = 3\alpha^2 -2\alpha + 1$$
  \textbf{Ответ: } $3\alpha^2 -2\alpha + 1$

  \item[\textbf{№2}]
  По условию:
  $$g = x_2^4x_3^5 + 2x_1x_2^4x_3 + x_1^2x_2^2, \quad f = x_2^4x_3 - 2x_1x_2x_3^2 + x_1x_2^2$$
  Так как
  $$LT(g) = x_1^2 x_2^2$$
  то 
  $$g \to g-x_1f = x_2^4x_3^5 + 2x_1x_2^4x_3 + x_1^2x_2^2 - (x_1x_2^4x_3 - 2x_1^2x_2x_3^2 + x_1^2x_2^2) = $$
  $$=x_2^4x_3^5 + x_1x_2^4x_3 + 2x_1^2x_2x_3^2 = g_1 $$
  Так как 
  $$LT(g_1) = 2x_1^2x_2x_3^2$$
  то
  $$g_1 \to g_1 + x_1f = x_2^4x_3^5 + x_1x_2^4x_3 + 2x_1^2x_2x_3^2  + x_1(x_2^4x_3 - 2x_1x_2x_3^2 + x_1x_2^2) = $$
  $$=x_2^4x_3^5 + x_1x_2^4x_3 + 2x_1^2x_2x_3^2  + x_1x_2^4x_3 - 2x_1^2x_2x_3^2 + x_1^2x_2^2 = $$
  $$ = x_2^4x_3^5 + 2x_1x_2^4x_3 + x_1^2x_2^2 = g$$
  Следовательно, мы вернулись к $g$, значит:
  $$r = g_1 = x_2^4x_3^5 + x_1x_2^4x_3 + 2x_1^2x_2x_3^2$$
  \textbf{Ответ: } $x_2^4x_3^5 + x_1x_2^4x_3 + 2x_1^2x_2x_3^2$\\

  \item[\textbf{№3}] Проверим, является ли множество $\{f_1, f_2, f_3\}$ системой Грёбнера:
  $$f_1 =  2x_1x_2 \;+\;4x_1x_3\;+\;x_2x_3^2, \quad f_2 =4x_1x_3^2\;+\;x_2x_3^3\;+\;4, \quad f_3 = x_2^2x_3^3\;+\;4x_2\;+\;8x_3 $$
  Возьмём $S$ многочлен для $f_2$ и $f_3$:
  $$\text{НОК}(x_1x_3^2,\;x_2^2x_3^3)=x_1x_2^2x_3^3 
  \implies S_{23}=\frac{x_2^2x_3}{4}\,f_2 \;-\;x_1\,f_3 = $$
  $$=\frac{x_2^2x_3}{4}(4x_1x_3^2+x_2x_3^3+4)
- x_1\bigl(x_2^2x_3^3+4x_2+8x_3\bigr) = $$
  $$
  =\bigl(x_1x_2^2x_3^3 - x_1x_2^2x_3^3\bigr)
  + \frac14\,x_2^3x_3^4
  + x_2^2x_3
  - 4x_1x_2
  - 8x_1x_3
  \;=$$
  $$=\;
  \frac14\,x_2^3x_3^4
  + x_2^2x_3
  - 4x_1x_2
  - 8x_1x_3 = g_1
  $$
  Так как $LT(g_1) = -4x_1x_2$, то редуцируем $g_1$ с помощью $f_1$:
  $$g_1 \to g_1 +2f_1 =  \frac{1}{4}\,x_2^3x_3^4
  + x_2^2x_3
  + 2\,x_2x_3^2 = g_2$$
  Так как $LT(g_2) = x_2^2x_3$, то редуцируем $g_2$ с помощью $f_3$:
  $$g_2 \to g_2 - \f{1}{4}x_2x_3f_3 = 0$$
  Таким образом, остаток \(S_{23}\) равен нулю

  Возьмём $S$ многочлен для $f_1$ и $f_2$:
  \[S_{12}
  =\frac{x_3^2}{2}\,f_1 \;-\;\frac{x_2}{4}\,f_2
  \]
  \[
  \begin{aligned}
  S_{12}
  &=\frac{x_3^2}{2}\bigl(2x_1x_2+4x_1x_3+x_2x_3^2\bigr)
  \;-\;\frac{x_2}{4}\bigl(4x_1x_3^2+x_2x_3^3+4\bigr)\\
  &=\bigl(x_1x_2x_3^2 - x_1x_2x_3^2\bigr)
  +2x_1x_3^3
  +\frac14\,x_2^2x_3^4
  - x_2x_3
  \\
  &\;=\;2x_1x_3^3 + \tfrac14\,x_2^2x_3^4 - x_2x_3
  \;=\;g_1
  \end{aligned}
  \]
  Поскольку \(LT(g_1)=2x_1x_3^3\), редуцируем по \(f_2\):
  \[
  g_1 \;-\;\tfrac12\,f_2
  =\bigl(2x_1x_3^3 -2x_1x_3^3\bigr)
  +\tfrac14\,x_2^2x_3^4
  - x_2x_3
  -2
  \;=\;\tfrac14\,x_2^2x_3^4 - x_2x_3 -2
  \;=\;g_2
  \]
  Теперь \(LT(g_2)=\tfrac14x_2^2x_3^4\), редуцируем по \(f_3\):
  \[
  g_2 \;-\;\tfrac{1}{4}x_3\,f_3
  =\bigl(\tfrac14x_2^2x_3^4 - \tfrac14x_2^2x_3^4\bigr)
  - x_2x_3
  -2
  +2
  \;=\;-x_2x_3 +0
  \;=\;g_3
  \]
  Наконец, \(LT(g_3)=-x_2x_3\), снова редуцируем по \(f_1\):
  \[
  g_3 \;+\; \bigl(-\tfrac12x_3\bigr)\,f_1
  = \bigl(-x_2x_3 + x_2x_3\bigr)
  +0
  \;=\;0
  \]
  Таким образом, остаток \(S_{12}\) равен нулю
  \[
  LT(f_1)=x_1x_2,\quad LT(f_3)=x_2^2x_3^3,
  \]
  \[
  \text{НОК}(x_1x_2,\;x_2^2x_3^3)=x_1x_2^2x_3^3
  \;\;\Longrightarrow\;\;
  S_{13}
  =\frac{x_2x_3^3}{2}\,f_1 \;-\;x_1\,f_3
  \]
  \[
  \begin{aligned}
  S_{13}
  &=\frac{x_2x_3^3}{2}\bigl(2x_1x_2+4x_1x_3+x_2x_3^2\bigr)
  - x_1\bigl(x_2^2x_3^3+4x_2+8x_3\bigr)\\
  &=\bigl(x_1x_2^2x_3^3 - x_1x_2^2x_3^3\bigr)
  +2x_1x_2x_3^4
  +\tfrac12\,x_2^2x_3^5
  -4x_1x_2
  -8x_1x_3
  \\
  &\;=\;2x_1x_2x_3^4 + \tfrac12\,x_2^2x_3^5 -4x_1x_2 -8x_1x_3
  \;=\;h_1
  \end{aligned}
  \]
  Редукция по \(f_1\) (\(LT(h_1)=2x_1x_2x_3^4\)):
  \[
  h_1 \;-\;x_3^3\,f_1
  =\bigl(2x_1x_2x_3^4 -2x_1x_2x_3^4\bigr)
  +\tfrac12\,x_2^2x_3^5
  -8x_1x_3
  +4x_1x_3
  \;=\;\tfrac12\,x_2^2x_3^5 -4x_1x_3
  \;=\;h_2
  \]
  Редукция по \(f_2\) (\(LT(h_2)=\tfrac12x_2^2x_3^5\)):
  \[
  h_2 \;-\;\tfrac{1}{8}x_2\,f_2
  =\bigl(\tfrac12x_2^2x_3^5 - \tfrac12x_2^2x_3^5\bigr)
  -4x_1x_3
  - \tfrac18x_2^2x_3^4
  +0
  \;=\;-4x_1x_3 - \tfrac18x_2^2x_3^4
  \;=\;h_3
  \]
  Редукция по \(f_3\) (\(LT(h_3)=-4x_1x_3\)):
  \[
  h_3 \;+\;\tfrac12x_1\,f_3
  =\bigl(-4x_1x_3 +4x_1x_3\bigr)
  - \tfrac18x_2^2x_3^4
  +0
  \;=\;- \tfrac18x_2^2x_3^4
  \;=\;h_4
  \]
  И снова по \(f_3\) (\(LT(h_4)=-\tfrac18x_2^2x_3^4\)):
  \[
  h_4 \;+\;\tfrac{1}{8}x_3\,f_3
  =\bigl(- \tfrac18x_2^2x_3^4 + \tfrac18x_2^2x_3^4\bigr)
  \;=\;0
  \]
  Таким образом, остаток \(S_{13}\) равен нулю

  Получается, что любой $S$ многочлен редуцируем к нулю, а значит это система Грёбнера.

  \item[\textbf{№4}] 
  \begin{enumerate}
    \item[$(\Rightarrow)$]Пусть \( F \) — система Грёбнера. 
    
    По определению, идеал старших членов \( \langle \text{LT}(I) \rangle \) порождается старшими членами элементов \( F \), то есть 
    \[ \langle \text{LT}(F) \rangle = \langle \text{LT}(I) \rangle \]
    Рассмотрим множество старших членов \( \{\text{LT}(f) \mid f \in F\} \). 
    В этом множестве должен существовать элемент с минимальным старшим членом 
    (по лексикографическому порядку). 
    
    Обозначим такой элемент \( \text{LT}(f) \), где \( f \in F \). 
    
    Так как \( \text{LT}(f) \) делит все остальные \( \text{LT}(g) \) для \( g \in F \)
    (иначе \( \text{LT}(g) \) не принадлежал бы идеалу, порождённому \( \text{LT}(f) \))
    то сам многочлен \( f \) делит каждый \( g \in F \).
    
    Это следует из того, что старший член \( f \) делит старший член \( g \), 
    а остальные члены \( g \) могут быть редуцированы с помощью \( f \).\\

    \item[$(\Leftarrow)$]
    Пусть существует \( f \in F \), который делит любой \( g \in F \). 
    
    Тогда старший член \( \text{LT}(f) \) делит \( \text{LT}(g) \) для всех \( g \in F \).
    
    Следовательно, идеал старших членов \( \langle \text{LT}(F) \rangle \) 
    порождается \( \text{LT}(f) \). 
    
    Это означает, что \( \langle \text{LT}(F) \rangle = \langle \text{LT}(I) \rangle \),
      так как все старшие члены элементов \( F \) уже содержатся в идеале, 
      порождённом \( \text{LT}(f) \).
      По определению, \( F \) является системой Грёбнера.
  \end{enumerate}
  Таким образом, \( F \) — система Грёбнера тогда и только тогда, когда существует \( f \in F \), делящий все элементы \( F \).


\end{enumerate}
\end{document}