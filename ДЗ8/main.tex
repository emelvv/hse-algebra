\documentclass[a4paper]{article}
\usepackage{setspace}
\usepackage[utf8]{inputenc}
\usepackage[russian]{babel}
\usepackage[12pt]{extsizes}
\usepackage{mathtools}
\usepackage{graphicx}
\usepackage{fancyhdr}
\usepackage{amssymb}
\usepackage{amsmath, amsfonts, amssymb, amsthm, mathtools}
\usepackage{tikz}

\usetikzlibrary{positioning}
\setstretch{1.3}

\newcommand{\mat}[1]{\begin{pmatrix} #1 \end{pmatrix}}
\newcommand{\vmat}[1]{\begin{vmatrix} #1 \end{vmatrix}}
\renewcommand{\f}[2]{\frac{#1}{#2}}
\newcommand{\dspace}{\space\space}
\newcommand{\s}[2]{\sum\limits_{#1}^{#2}}
\newcommand{\mul}[2]{\prod_{#1}^{#2}}
\newcommand{\sq}[1]{\left[ {#1} \right]}
\newcommand{\gath}[1]{\left[ \begin{array}{@{}l@{}} #1 \end{array} \right.}
\newcommand{\case}[1]{\begin{cases} #1 \end{cases}}
\newcommand{\ts}{\text{\space}}
\newcommand{\lm}[1]{\underset{#1}{\lim}}
\newcommand{\suplm}[1]{\underset{#1}{\overline{\lim}}}
\newcommand{\inflm}[1]{\underset{#1}{\underline{\lim}}}
\newcommand{\Ker}[1]{\operatorname{Ker}}

\renewcommand{\phi}{\varphi}
\newcommand{\lr}{\Leftrightarrow}
\renewcommand{\l}{\left(}
\renewcommand{\r}{\right)}
\newcommand{\rr}{\rightarrow}
\renewcommand{\geq}{\geqslant}
\renewcommand{\leq}{\leqslant}
\newcommand{\RR}{\mathbb{R}}
\newcommand{\CC}{\mathbb{C}}
\newcommand{\QQ}{\mathbb{Q}}
\newcommand{\ZZ}{\mathbb{Z}}
\newcommand{\VV}{\mathbb{V}}
\newcommand{\NN}{\mathbb{N}}
\newcommand{\OO}{\underline{O}}
\newcommand{\oo}{\overline{o}}
\renewcommand{\Ker}{\operatorname{Ker}}
\renewcommand{\Im}{\operatorname{Im}}
\newcommand{\vol}{\text{vol}}
\newcommand{\Vol}{\text{Vol}}

\DeclarePairedDelimiter\abs{\lvert}{\rvert} %
\makeatletter                               % \abs{}
\let\oldabs\abs                             %
\def\abs{\@ifstar{\oldabs}{\oldabs*}}       %

\begin{document}

\section*{Домашнее задание на 08.06 (Алгебра)}
 {\large Емельянов Владимир, ПМИ гр №247}\\\\
\begin{enumerate}
  \item[\textbf{№1}]Чтобы избавиться от иррациональности в знаменателе, положим
  $$
  x = \sqrt[3]{7},\quad \text{тогда } \sqrt[3]{49} = x^2,\; x^3 = 7
  $$
  Наша дробь превращается в
  $$
  \frac{\,1 + 55x \;-\; 8x^2\,}{\,1 \;-\; 2x \;-\; 4x^2\,}
  $$
  Обозначим знаменатель
  $$
  D(x) = 1 - 2x - 4x^2
  $$
  Найдём многочлен
  $$
  P(x) = a + b\,x + c\,x^2
  $$
  такой, что $D(x)\,P(x)$ — просто рациональное число. Тогда
  $$
  \frac{1 + 55x - 8x^2}{D(x)} \;=\; \frac{(1 + 55x - 8x^2)\,P(x)}{D(x)\,P(x)}
  $$
  Распишем
  $$
  P(x) \;=\; a + b\,x + c\,x^2,
  \qquad 
  D(x)\,P(x) \;=\; (1 - 2x - 4x^2)\,(a + b\,x + c\,x^2)
  $$
  При перемножении и сведении всех степеней $x$ к остаткам по модулю $x^3 - 7 = 0$ (то есть используя $x^3 = 7$ и $x^4 = 7x$) получаем:
  $$
  D(x)\,P(x) \;=\; \bigl(a - 28\,b - 14\,c\bigr)
  \;+\;\bigl(-2a + b - 28c\bigr)\,x 
  \;+\;\bigl(-4a - 2b + c\bigr)\,x^2
  $$
  Чтобы в этом произведении не было членов с $x$ и $x^2$, решаем систему:
  $$
  \begin{cases}
  -2a + b - 28c = 0,\\
  -4a - 2b + c = 0.
  \end{cases} \implies b = \frac{-114a}{55}, \quad c = \frac{-8a}{55} 
  $$
  Пусть $a = 55$, тогда $b = -114$, $c = -8$
      
  Тогда
  $$
  P(x) \;=\; 55 \;-\; 114\,x \;-\; 8\,x^2
  $$
  $$
  D(x)P(x) = 3359
  $$
  То есть:
  $$
  \frac{\,1 + 55x \;-\; 8x^2\,}{\,1 \;-\; 2x \;-\; 4x^2\,} =
  \frac{(\,1 + 55x \;-\; 8x^2\,)P(x)}{3359}
  $$
  Вычислим:
  $$(\,1 + 55x \;-\; 8x^2\,)P(x) = (\,1 + 55x \;-\; 8x^2\,)(\; 55 \;-\; 114\,x \;-\; 8\,x^2) = $$
  $$=3359 + 3359x - 6718x^2$$
  Получаем:
  $$\frac{\,1 + 55x \;-\; 8x^2\,}{\,1 \;-\; 2x \;-\; 4x^2\,} = 
  \f{3359 + 3359x - 6718x^2}{3359} = 1 + x - 2x^2 = 1 + \sqrt[3]{7} -2\sqrt[3]{49}$$
  \textbf{Ответ: } $1 + \sqrt[3]{7} -2\sqrt[3]{49}$\\

  \item[\textbf{№2}]Рассмотрим число
  $$
  x \;=\; \sqrt{5} \;-\; \sqrt{3} \;+\; 1.
  $$
  Пусть 
  $$
  y = x - 1 = \sqrt{5} - \sqrt{3}.
  $$
  Найдём минимальный многочлен для $y$ над $\mathbb{Q}$.

  Поскольку все $\mathbb{Q}$-автоморфизмы поля $\mathbb{Q}(\sqrt5,\sqrt3)$ 
  независимо меняют знаки у $\sqrt5$ и $\sqrt3$, число $y=\sqrt5-\sqrt3$ 
  имеет ровно четыре значения $\pm\sqrt5\pm\sqrt3$, и его минимальный многочлен:
  $$(y-(\sqrt{5} - \sqrt{3}))(y-(\sqrt{5} + \sqrt{3}))(y-(-\sqrt{5} - \sqrt{3}))
  (y-(-\sqrt{5} + \sqrt{3})) = $$
  После раскрытия скобок получаем:
  $$y^4 - 16y^2 + 4$$
  Искомый минимальный многочлен для $x$:
  $$x^4 - 4x^3 - 10x^2 + 28x - 11$$
  \textbf{Ответ: } $x^4 - 4x^3 - 10x^2 + 28x - 11$\\


  \item[\textbf{№3}]

  \end{enumerate}
\end{document}