\documentclass[a4paper]{article}
\usepackage{setspace}
\usepackage[T2A]{fontenc} %
\usepackage[utf8]{inputenc} % подключение русского языка
\usepackage[russian]{babel} %
\usepackage[12pt]{extsizes}
\usepackage{mathtools}
\usepackage{graphicx}
\usepackage{fancyhdr}
\usepackage{amssymb}
\usepackage{amsmath, amsfonts, amssymb, amsthm, mathtools}
\usepackage{tikz}

\usetikzlibrary{positioning}
\setstretch{1.3}

\newcommand{\mat}[1]{\begin{pmatrix} #1 \end{pmatrix}}
\renewcommand{\det}[1]{\begin{vmatrix} #1 \end{vmatrix}}
\renewcommand{\f}[2]{\frac{#1}{#2}}
\newcommand{\dspace}{\space\space}
\newcommand{\s}[2]{\sum\limits_{#1}^{#2}}
\newcommand{\mul}[2]{\prod_{#1}^{#2}}
\newcommand{\sq}[1]{\left[ {#1} \right]}
\newcommand{\gath}[1]{\left[ \begin{array}{@{}l@{}} #1 \end{array} \right.}
\newcommand{\case}[1]{\begin{cases} #1 \end{cases}}
\newcommand{\ts}{\text{\space}}
\newcommand{\lm}[1]{\underset{#1}{\lim}}
\newcommand{\suplm}[1]{\underset{#1}{\overline{\lim}}}
\newcommand{\inflm}[1]{\underset{#1}{\underline{\lim}}}
\newcommand{\Ker}{\operatorname{Ker}}
\renewcommand{\Im}{\operatorname{Im}}

\renewcommand{\phi}{\varphi}
\newcommand{\lr}{\Leftrightarrow}
\renewcommand{\r}{\Rightarrow}
\newcommand{\rr}{\rightarrow}
\renewcommand{\geq}{\geqslant}
\renewcommand{\leq}{\leqslant}
\newcommand{\RR}{\mathbb{R}}
\newcommand{\CC}{\mathbb{C}}
\newcommand{\QQ}{\mathbb{Q}}
\newcommand{\ZZ}{\mathbb{Z}}
\newcommand{\VV}{\mathbb{V}}
\newcommand{\NN}{\mathbb{N}}
\newcommand{\OO}{\underline{O}}
\newcommand{\oo}{\overline{o}}
\newcommand{\divides}{\;|\;}
\newcommand{\leg}[2]{\left(\f{#1}{#2}\right)}

\DeclarePairedDelimiter\abs{\lvert}{\rvert} %
\makeatletter                               % \abs{}
\let\oldabs\abs                             %
\def\abs{\@ifstar{\oldabs}{\oldabs*}}       %

\begin{document}

\section*{Домашнее задание на 24.04 (Алгебра)}
 {\large Емельянов Владимир, ПМИ гр №247}\\\\
\begin{enumerate}
    \item[\textbf{№1}]Элементы из $\ZZ_2 \times \ZZ_{10} \times \ZZ_{25}$
    \begin{enumerate}
        \item[1)]
        \underline{порядка 2:}
        $$(1, 0, 0)\quad (1, 5, 0) \quad (0, 5, 0)$$
        \textbf{всего: }$3$

        \item[2)]
        \underline{порядка 5:}

        все тройки порядка 5 имеют вид:
        $$(0, 2m, 5n) \quad  \forall m \in [0, 4], n \in [0, 4]$$
        То есть всего элементов порядка 5:
        $$5 \times 5 - 1 = 24$$
        (вычитаем тройку $(0, 0, 0)$)

        \item[3)]
        \underline{порядка 10:}

        все тройки порядка 10 имеют вид:
        \begin{enumerate}
            \item[a)] на первом месте 0:
            $$(0, 1|3|7|9, 0|5|10|15|20) + (0, 5, 5|10|15|20) = 4\cdot 5 + 4 = 24$$
            \item[b)] на первом месте 1:
            $$(1, 1|2|3|4|6|8|7|9, 0|5|10|15|20) + (1, 0|1|2|3|4|5|6|8|7|9, 5|10|15|20)-$$
            $$-(1, 1|2|3|4|6|8|7|9, 5|10|15|20) = 8\cdot 5 + 10\cdot 4 -8\cdot 4 = 48$$
        \end{enumerate}
        Всего элементов:
        $$48+24 = 72$$

        \item[4)]
        \underline{порядка 25:}

        все тройки порядка 25 имеют вид:
        $$(0, 0|2|4|5|8, 1|2|3|4|6|...\small{(\text{не кратные } 5)}) = 5 \cdot 20 = 100$$

    \end{enumerate}

    \textbf{Ответ: }$3, 24, 72, 100$\\

    \item[\textbf{№2}] Так как:
    $$63 = 3^2 \cdot 7$$
    То по классификации конечных абелевых групп $A$ раскладывается:
    \[
     A\;\simeq\;Z_{3^{k_1}}\times\cdots\times Z_{3^{k_r}}\times Z_{7^{\ell_1}}\times\cdots\times Z_{7^{\ell_s}},
    \]
    где сумма всех \(3\)-показателей \(k_i\) равна 2, а сумма всех \(7\)-показателей \(\ell_j\) равна 1
    
    Однако если бы в разложении встретился сомножитель \(Z_{9}\times Z_7\), то по теореме 3 (для \(n=9\cdot7\), \(\gcd(9,7)=1\))  
    мы получили бы циклическую группу \(Z_{63}\).
    Поскольку \(A\) по условию — нециклическая, единственное возможное разложение:  
    \[
      A\;\simeq\;Z_{3}\times Z_{3}\times Z_{7}.
    \]
    \begin{enumerate}
        \item[1)]
        Любая подгруппа порядка 3 должна лежать в сомножителе \(Z_3\times Z_3\).  
        
        — В \(Z_3\times Z_3\) всего \(3^2=9\) элемента, из них один нейтральный, а остальные 8 имеют порядок 3.  
        
        — Циклическая подгруппа порядка 3 содержит ровно 2 элемента порядка 3 .  
        
        Поэтому число различных подгрупп порядка 3 равно  
        \[
        \frac{8}{2} \;=\;4.
        \]

        \item[2)]Подгруппа порядка 21 должна содержать одновременно элемент порядка 3 и элемент порядка 7.  
        
        — В \(Z_7\) существует ровно одна подгруппа порядка 7 (сама \(Z_7\)).  
        
        — В \(Z_3\times Z_3\) — 4 подгруппы порядка 3.  

        Пусть \(H_3\) — любая из 4 подгрупп порядка 3, а \(H_7=Z_7\). Тогда в абелевой группе \(A\) произведение \(H_3H_7\) есть подгруппа (т.к. \(H_3\cap H_7=\{e\}\)) и по теореме Лагранжа её порядок  
        \(\;|H_3H_7|=|H_3|\cdot|H_7|=3\cdot7=21\).  
        
        Разные \(H_3\) дают разные произведения \(H_3H_7\), значит, всего подгрупп порядка 21 тоже 4.

    \end{enumerate}
            

\end{enumerate}
\end{document}