\documentclass[a4paper]{article}
\usepackage{setspace}
\usepackage[utf8]{inputenc}
\usepackage[russian]{babel}
\usepackage[12pt]{extsizes}
\usepackage{mathtools}
\usepackage{graphicx}
\usepackage{fancyhdr}
\usepackage{amssymb}
\usepackage{amsmath, amsfonts, amssymb, amsthm, mathtools}
\usepackage{tikz}

\usetikzlibrary{positioning}
\setstretch{1.3}

\newcommand{\mat}[1]{\begin{pmatrix} #1 \end{pmatrix}}
\newcommand{\vmat}[1]{\begin{vmatrix} #1 \end{vmatrix}}
\renewcommand{\f}[2]{\frac{#1}{#2}}
\newcommand{\dspace}{\space\space}
\newcommand{\s}[2]{\sum\limits_{#1}^{#2}}
\newcommand{\mul}[2]{\prod_{#1}^{#2}}
\newcommand{\sq}[1]{\left[ {#1} \right]}
\newcommand{\gath}[1]{\left[ \begin{array}{@{}l@{}} #1 \end{array} \right.}
\newcommand{\case}[1]{\begin{cases} #1 \end{cases}}
\newcommand{\ts}{\text{\space}}
\newcommand{\lm}[1]{\underset{#1}{\lim}}
\newcommand{\suplm}[1]{\underset{#1}{\overline{\lim}}}
\newcommand{\inflm}[1]{\underset{#1}{\underline{\lim}}}
\newcommand{\Ker}[1]{\operatorname{Ker}}

\renewcommand{\phi}{\varphi}
\newcommand{\lr}{\Leftrightarrow}
\renewcommand{\l}{\left(}
\renewcommand{\r}{\right)}
\newcommand{\rr}{\rightarrow}
\renewcommand{\geq}{\geqslant}
\renewcommand{\leq}{\leqslant}
\newcommand{\RR}{\mathbb{R}}
\newcommand{\CC}{\mathbb{C}}
\newcommand{\QQ}{\mathbb{Q}}
\newcommand{\ZZ}{\mathbb{Z}}
\newcommand{\VV}{\mathbb{V}}
\newcommand{\NN}{\mathbb{N}}
\newcommand{\OO}{\underline{O}}
\newcommand{\oo}{\overline{o}}
\renewcommand{\Ker}{\operatorname{Ker}}
\renewcommand{\Im}{\operatorname{Im}}
\newcommand{\vol}{\text{vol}}
\newcommand{\Vol}{\text{Vol}}

\DeclarePairedDelimiter\abs{\lvert}{\rvert} %
\makeatletter                               % \abs{}
\let\oldabs\abs                             %
\def\abs{\@ifstar{\oldabs}{\oldabs*}}       %

\begin{document}

\section*{Домашнее задание на 03.06 (Алгебра)}
 {\large Емельянов Владимир, ПМИ гр №247}\\\\
\begin{enumerate}
  \item[\textbf{№1}]Пусть
  \[ g_1 = x^2 y + 2z^2, \quad g_2 = y^2 - yz \]
  \[ \text{L}(g_1) = x^2 y, \quad \text{L}(g_2) = y^2, \quad
    \text{lcm}(x^2 y, y^2) = x^2 y^2, \]
  \[ S(g_1, g_2) = \frac{x^2 y^2}{x^2 y} g_1 - \frac{x^2 y^2}{y^2} g_2 =
    y g_1 - x^2 g_2 = y(x^2 y + 2z^2) - x^2(y^2 - yz) =\]
  \[ x^2 y^2 + 2y z^2 - x^2 y^2 +
    x^2 y z = x^2 y z + 2y z^2 \]
  так как:
  \[ z \cdot g_1 = z(x^2 y + 2z^2) = x^2 y z + 2z^3 \]
  вычтем:
  \[ (x^2 y z + 2y z^2) - (x^2 y z + 2z^3) = 2y z^2 - 2z^3 \]
  Добавляем \( g_3 = y z^2 - z^3 \)

  Вычисляем \( S(g_1, g_3) \):
  \[ \text{L}(g_1) = x^2 y, \quad \text{L}(g_3) =
    y z^2, \quad \text{lcm}(x^2 y, y z^2) = x^2 y z^2, \]
  \[ S(g_1, g_3) = \frac{x^2 y z^2}{x^2 y} g_1 - \frac{x^2 y z^2}{y z^2} g_3 \]
  \[= z^2 g_1 - x^2 g_3 = z^2(x^2 y + 2z^2) - x^2(y z^2 - z^3) =
    x^2 y z^2 + 2z^4 - x^2 y z^2 + x^2 z^3 = x^2 z^3 + 2z^4 \]
  старший моном \( x^2 z^3 \) не делится ни на один старший моном базиса, поэтому добавляем \( g_4 = x^2 z^3 + 2z^4 \)

  Теперь базис: \( \{g_1, g_2, g_3, g_4\} = \{x^2 y + 2z^2, y^2 - yz, y z^2 - z^3, x^2 z^3 + 2z^4\} \).

  Проверяем остальные S-многочлены:
  \begin{itemize}
    \item\( S(g_2, g_3) = 0 \),
    \item\( S(g_2, g_4) \) редуцируется к 0,
    \item\( S(g_3, g_4) \) редуцируется к 0,
    \item\( S(g_1, g_4) \) редуцируется к 0.
  \end{itemize}
  Все S-многочлены редуцируются к нулю, поэтому базис 
  Грёбнера идеала \( I \):
  \[ G = \{x^2 y + 2z^2, y^2 - yz, y z^2 - z^3, x^2 z^3 + 2z^4\}. \]

  Теперь редуцируем \( f = x^3 y^2 z + b x y z^3 \) на \( G \):  
  \[ x^3 z \cdot g_2 = x^3 z (y^2 - yz) = x^3 y^2 z - x^3 y z^2, \]  
  \[ f - x^3 z g_2 = (x^3 y^2 z + b x y z^3) - (x^3 y^2 z - x^3 y z^2) = b x y z^3 + x^3 y z^2. \]  
  Остаток \( r_1 = x^3 y z^2 + b x y z^3 \).  

  \[ x z^2 \cdot g_1 = x z^2 (x^2 y + 2z^2) = x^3 y z^2 + 2x z^4, \]  
  \[ r_1 - x z^2 g_1 = (x^3 y z^2 + b x y z^3) - (x^3 y z^2 + 2x z^4) = 
  b x y z^3 - 2x z^4 \]  
  Остаток \( r_2 = b x y z^3 - 2x z^4 \).  

  \[ b x z \cdot g_3 = b x z (y z^2 - z^3) = b x y z^3 - b x z^4, \]  
  \[ r_2 - b x z g_3 = (b x y z^3 - 2x z^4) - (b x y z^3 - b x z^4) = (b - 2) x z^4 \]  
  Остаток \( r_3 = (b - 2) x z^4 \).  

  Старший моном \( \text{L}(r_3) = x z^4 \) не делится ни на один старший моном базиса \( G \), так как:  
  \begin{itemize}
  \item  \( \text{L}(g_1) = x^2 y \) не делит (степень \( x \) выше)  
  \item  \( \text{L}(g_2) = y^2 \) не делит (отсутствует \( y \))
  \item  \( \text{L}(g_3) = y z^2 \) не делит (отсутствует \( y \))  
  \item  \( \text{L}(g_4) = x^2 z^3 \) не делит (степень \( x \) выше)
  \end{itemize}

  Поэтому остаток \[ r_3 = (b - 2) x z^4 \]

  Остаток равен нулю тогда и только тогда, когда \( b - 2 = 0 \), то есть \( b = 2 \). При \( b = 2 \) многочлен \( f \) принадлежит идеалу \( I \), так как редуцируется к нулю относительно базиса Грёбнера. При других значениях \( b \) остаток ненулевой, поэтому \( f \notin I \).

  Таким образом, единственное значение параметра \( b \), при котором \( f \in I \), это \( b = 2 \).

  \textbf{Ответ: } $b = 2$\\

  \item[\textbf{№2}]Пусть
  \[
  f_1 = y^3 + 3xy, \quad f_2 = xy^2 + 2x^2 + y, \quad f_3 = x^2y - y^2.
  \]
  Старшие мономы при порядке \( x \succ y \):
  \begin{itemize}
    \item \( \text{LM}(f_1) = xy \) (так как \( xy \) содержит \( x \), а \( y^3 \) 
    не содержит \( x \), и \( x \succ y \)).
    \item \( \text{LM}(f_2) = x^2 \) (моном \( x^2 \) старше \( xy^2 \) и \( y \)).
    \item \( \text{LM}(f_3) = x^2y \) (моном \( x^2y \) старше \( y^2 \)).
  \end{itemize}

  Вычисляем s-многочлены

  \begin{itemize}
    \item  $S(f_1, f_2)$:
      \[
      S(f_1, f_2) = \frac{\text{lcm}(xy, x^2)}{xy} f_1 - 
      \frac{\text{lcm}(xy, x^2)}{x^2} f_2 = x f_1 - y f_2
      \]
      Подставляем:
      \[
      x f_1 = x(y^3 + 3xy) = xy^3 + 3x^2y, \quad y f_2 = y(xy^2 + 2x^2 + y) = 
      xy^3 + 2x^2y + y^2
      \]
      \[
      S(f_1, f_2) = (xy^3 + 3x^2y) - (xy^3 + 2x^2y + y^2) = x^2y - y^2 = f_3
      \]
      Остаток 0, так как \( f_3 \) уже в базисе.
    
    \item  $S(f_1, f_3)$:
      \[
      S(f_1, f_3) = \frac{\text{lcm}(xy, x^2y)}{xy} f_1 -
       \frac{\text{lcm}(xy, x^2y)}{x^2y} f_3 = x f_1 - f_3
      \]
      Подставляем:
      \[
      x f_1 = x(y^3 + 3xy) = xy^3 + 3x^2y, \quad f_3 = x^2y - y^2,
      \]
      \[
      S(f_1, f_3) = (xy^3 + 3x^2y) - (x^2y - y^2) = xy^3 + 2x^2y + y^2
      \]
      Редуцируем по базису \( \{f_1, f_2, f_3\} \):
      
      Старший моном \( 2x^2y \) делится на \( \text{LM}(f_3) = x^2y \) с частным 2:
        \[
        2 f_3 = 2(x^2y - y^2) = 2x^2y - 2y^2,
        \]
        \[
        (xy^3 + 2x^2y + y^2) - (2x^2y - 2y^2) = xy^3 + 3y^2
        \]
       Старший моном \( xy^3 \) делится на \( \text{LM}(f_1) = xy \) с частным
       \( \frac{1}{3} y^2 \):
        \[
        \frac{1}{3} y^2 f_1 = \frac{1}{3} y^2 (y^3 + 3xy) = \frac{1}{3} y^5 + xy^3,
        \]
        \[
        (xy^3 + 3y^2) - \left( \frac{1}{3} y^5 + xy^3 \right) = -\frac{1}{3} y^5 + 3y^2
        \]
       Моном \( -\frac{1}{3} y^5 \) не делится на старшие мономы базиса. Добавляем новый
        многочлен \( f_4 = y^5 - 9y^2 \) (умножив остаток на -3 для удобства).
    
      Теперь базис: \( \{f_1, f_2, f_3, f_4\} \).

    \item$S(f_2, f_3)$:
    \[
    S(f_2, f_3) = \frac{\text{lcm}(x^2, x^2y)}{x^2} f_2 -
     \frac{\text{lcm}(x^2, x^2y)}{x^2y} f_3 = y f_2 - f_3
    \]
    Подставляем:
    \[
    y f_2 = y(xy^2 + 2x^2 + y) = xy^3 + 2x^2y + y^2, \quad f_3 = x^2y - y^2,
    \]
    \[
    S(f_2, f_3) = (xy^3 + 2x^2y + y^2) - (x^2y - y^2) = xy^3 + x^2y + 2y^2
    \]
    Редуцируем по базису \( \{f_1, f_2, f_3, f_4\} \):
    
    Старший моном \( x^2y \) делится на \( \text{LM}(f_3) = x^2y \) с частным 1:
      \[
      f_3 = x^2y - y^2, \quad S - f_3 = (xy^3 + x^2y + 2y^2) - 
      (x^2y - y^2) = xy^3 + 3y^2
      \]
    Как и ранее, редуцируется до \( -\frac{1}{3} y^5 + 3y^2 \), который редуцируется к
     0 с помощью \( f_4 \).
  \end{itemize}
  Вычисляем остальные S-многочлены $(S(f_1, f_4), S(f_2, f_4), S(f_3, f_4))$, и
   все редуцируются к 0. Таким образом, базис Грёбнера: 
   \( \{f_1, f_2, f_3, f_4\} = \{y^3 + 3xy,  xy^2 + 2x^2 + y,  x^2y - y^2,  y^5 - 9y^2\} \).

  Для получения минимального редуцированного базиса:
  \begin{enumerate}
    \item Удаляем многочлены, старшие мономы которых делятся на старшие мономы других многочленов.
    \item Делаем старшие коэффициенты равными 1.
    \item Редуцируем каждый многочлен по остальным.
  \end{enumerate}

  
  Старшие мономы:
  \begin{itemize}
    \item $\text{LM}(f_3) = x^2y$ делится на $\text{LM}(f_2) = x^2$, удаляем $f_3$.
    \item $\text{LM}(f_4) = y^5$ не делится на другие, оставляем.
    \item $\text{LM}(f_1) = xy$ не делится на $x^2$ или $y^5$, оставляем.
    \item $\text{LM}(f_2) = x^2$ не делится на $xy$ или $y^5$, оставляем.
  \end{itemize}
  
  Базис после минимизации: \( \{f_1, f_2, f_4\} = \{y^3 + 3xy,  xy^2 + 2x^2 + y,  y^5 - 9y^2\} \).

  Делаем старшие коэффициенты равными 1:
  \[
  g_1 = \frac{1}{3} f_1 = \frac{1}{3} y^3 + xy, \quad g_2 = \frac{1}{2} f_2 =
   \frac{1}{2} xy^2 + x^2 + \frac{1}{2} y, \quad g_4 = f_4 = y^5 - 9y^2
  \]

  Теперь редуцируем каждый полином по остальным
  \begin{itemize}
    \item 
    Редукция $g_1$ по $\{g_2,\,g_3\}$
  $$
  g_1 = x\,y + \tfrac{1}{3}\,y^3
  $$
  Ведущий моном $x\,y$. $x^2$ (ведущий моном из $g_2$) не делит $x\,y$.
  
  $y^5$ из $g_3$ явно не делит ни $x\,y$, ни $y^3$.
  
  Следовательно, $g_1$  не изменяется : он уже «редуцирован» относительно
   $g_2$ и $g_3$.
  
  \item Редукция $g_2$ по $\{g_1,\,g_3\}$
  $$
  g_2 = x^2 \;+\; \tfrac12\,x\,y^2 \;+\; \tfrac12\,y
  $$
  Старший моном $x^2$. Ни $xy$ (из $g_1$), ни $y^5$ (из $g_3$) не делят $x^2$.
   Потому $x^2$ остаётся.
  
  Следующий по старшинству моном $\tfrac12\,x\,y^2$. Здесь $\mathrm{LM}(g_1)=x\,y$ 
  делит $x\,y^2$.
    $$
    \frac{x\,y^2}{\,x\,y\,} = y
    $$
    Значит, есть что вычесть:
    $$
    y \cdot g_1 
    = y\Bigl(x\,y + \tfrac13\,y^3\Bigr)
    = x\,y^2 \;+\; \tfrac13\,y^4
    $$
    Чтобы убрать ровно $\tfrac12\,x\,y^2$, нам нужно взять $\tfrac12$ от этого:
    $$
    \frac12 \bigl(y\,g_1\bigr) 
    = \frac12\,x\,y^2 \;+\; \frac{1}{6}\,y^4.
    $$
    Поэтому вычитаем из $g_2$ именно $\tfrac12\,y\,g_1$:
    $$
    g_2 
    - \frac12\,y\,g_1
    \;=\;
    \bigl(x^2 + \tfrac12\,x\,y^2 + \tfrac12\,y\bigr)
    \;-\;\Bigl(\tfrac12\,x\,y^2 + \tfrac{1}{6}\,y^4\Bigr)
    \;=\;
    x^2 
    \;+\; \frac12\,y 
    \;-\; \frac{1}{6}\,y^4
    $$
    После этой вычиталки $\tfrac12\,x\,y^2$ исчез, однако вместо него появился
     моном $-\tfrac{1}{6}y^4$.
   Теперь новое невырожденное сочетание равно

    $$
    \tilde g_2 \;=\; x^2 \;+\; \frac12\,y \;-\; \frac{1}{6}\,y^4
    $$

    Проверим оставшиеся мономы:

     $x^2$ уже не делится ни на $xy$ (из $g_1$), ни на $y^5$ (из $g_3$).

     $-\tfrac{1}{6}y^4$. Здесь $y^5$ не делит $y^4$, а $xy$ не делит $y^4$.

     $\tfrac12\,y$ тем более не делится ни на $xy$, ни на $y^5$.

      Значит, $\tilde g_2$ больше не редуцируется, и мы присваиваем:

    $$
    g_2' 
    \;:=\;
    x^2 \;-\; \frac{1}{6}\,y^4 \;+\; \frac12\,y
    $$

    \item Редукция $g_3$ по $\{g_1,\,g_2'\}$
    $$
    g_3 = y^5 - 9\,y^2
    $$
    Ведущий моном $y^5$. Ни $x\,y$ (из $g_1$), ни $x^2$ (из $g_2'$) не делят $y^5$.
    
    Обратный моном $-9y^2$ тоже не делится ни на $xy$, ни на $x^2$. Поэтому $g_3$ остаётся без изменений.
  \end{itemize}

  Получаем редуцированный минимальный базис Грёбнера
  $$
  \bigl\{\,g_1,\;g_2',\;g_3\bigr\} \;=\;
  \Bigl\{
  \,x\,y + \tfrac{1}{3}\,y^3,\;
  x^2 \;-\; \tfrac{1}{6}\,y^4 \;+\; \tfrac{1}{2}\,y,\;
  y^5 \;-\; 9\,y^2
  \Bigr\}
  $$
  \textbf{Ответ: } $\Bigl\{
  \,x\,y + \tfrac{1}{3}\,y^3,\;
  x^2 \;-\; \tfrac{1}{6}\,y^4 \;+\; \tfrac{1}{2}\,y,\;
  y^5 \;-\; 9\,y^2
  \Bigr\}$
  \end{enumerate}
\end{document}