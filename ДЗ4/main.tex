\documentclass[a4paper]{article}
\usepackage{setspace}
\usepackage[T2A]{fontenc} %
\usepackage[utf8]{inputenc} % подключение русского языка
\usepackage[russian]{babel} %
\usepackage[12pt]{extsizes}
\usepackage{mathtools}
\usepackage{graphicx}
\usepackage{fancyhdr}
\usepackage{amssymb}
\usepackage{amsmath, amsfonts, amssymb, amsthm, mathtools}
\usepackage{tikz}

\usetikzlibrary{positioning}
\setstretch{1.3}

\newcommand{\mat}[1]{\begin{pmatrix} #1 \end{pmatrix}}
\renewcommand{\f}[2]{\frac{#1}{#2}}
\newcommand{\dspace}{\space\space}
\newcommand{\s}[2]{\sum\limits_{#1}^{#2}}
\newcommand{\mul}[2]{\prod_{#1}^{#2}}
\newcommand{\sq}[1]{\left[ {#1} \right]}
\newcommand{\gath}[1]{\left[ \begin{array}{@{}l@{}} #1 \end{array} \right.}
\newcommand{\case}[1]{\begin{cases} #1 \end{cases}}
\newcommand{\ts}{\text{\space}}
\newcommand{\lm}[1]{\underset{#1}{\lim}}
\newcommand{\suplm}[1]{\underset{#1}{\overline{\lim}}}
\newcommand{\inflm}[1]{\underset{#1}{\underline{\lim}}}
\newcommand{\Ker}{\operatorname{Ker}}
\renewcommand{\Im}{\operatorname{Im}}

\renewcommand{\phi}{\varphi}
\newcommand{\lr}{\Leftrightarrow}
\renewcommand{\r}{\Rightarrow}
\newcommand{\rr}{\rightarrow}
\renewcommand{\geq}{\geqslant}
\renewcommand{\leq}{\leqslant}
\newcommand{\RR}{\mathbb{R}}
\newcommand{\CC}{\mathbb{C}}
\newcommand{\QQ}{\mathbb{Q}}
\newcommand{\ZZ}{\mathbb{Z}}
\newcommand{\VV}{\mathbb{V}}
\newcommand{\NN}{\mathbb{N}}
\newcommand{\OO}{\underline{O}}
\newcommand{\oo}{\overline{o}}
\newcommand{\divides}{\;|\;}
\newcommand{\leg}[2]{\left(\f{#1}{#2}\right)}

\DeclarePairedDelimiter\abs{\lvert}{\rvert} %
\makeatletter                               % \abs{}
\let\oldabs\abs                             %
\def\abs{\@ifstar{\oldabs}{\oldabs*}}       %

\begin{document}

\section*{Домашнее задание на 2.05 (Алгебра)}
 {\large Емельянов Владимир, ПМИ гр №247}\\\\
\begin{enumerate}
    \item[\textbf{№1}]Мы знаем, что:
    $$A=\begin{pmatrix}a&0\\ b&c\end{pmatrix} \text{--- обратима }\lr \det A \neq 0 \lr ac \neq 0 \lr \case{a \neq 0\\c \neq 0} $$
    Значит все обратимые элементы имеют вид:
    $$\begin{pmatrix}a&0\\ b&c\end{pmatrix}, \quad \text{ где } a\neq 0 \text{ и } c \neq 0$$
    В кольце матриц \(A\) — нулевой делитель (левый или правый) тогда и только тогда, когда \(\det A=0\), то есть  
    \[
    ac = 0.
    \]
    \begin{enumerate}
        \item[1)]Если $a = 0$:
        
        \[A=\begin{pmatrix}0&0\\ b&c\end{pmatrix}\]  
        Возьмём, например,  
        \[
        X=\begin{pmatrix}c&0\\-b&0\end{pmatrix}\neq0.
        \]
        Тогда
        \[
        A\,X
        =\begin{pmatrix}0&0\\ b&c\end{pmatrix}
            \begin{pmatrix}c&0\\-b&0\end{pmatrix}
        =\begin{pmatrix}0&0\\ b\,c - c\,b & 0\end{pmatrix}
        =\begin{pmatrix}0&0\\0&0\end{pmatrix}.
        \]
        Значит, \(A\) — левый нулевой делитель.

        \item[1)]Если $c = 0$:
        \[A=\begin{pmatrix}a&0\\ b&0\end{pmatrix}\]   
        Можно взять  
        \[
        X=\begin{pmatrix}0&a\\0&-b\end{pmatrix}\neq0
        \]
        и проверить \(A X=0\) аналогично.
    \end{enumerate}

    Аналогично для правых делителей, если \(ac=0\), то существует ненулевое \(Y\) с \(YA=0\).

    Найдём все нильпотентные \(A=\begin{pmatrix}a&0\\ b&c\end{pmatrix}\)

    Вычислим  
    \[
    A^2
    = \begin{pmatrix}a&0\\b&c\end{pmatrix}
        \begin{pmatrix}a&0\\b&c\end{pmatrix}
    = \begin{pmatrix}a^2 & 0\\ b\,a + c\,b & c^2\end{pmatrix}
    = \begin{pmatrix}a^2 & 0\\ b\,(a+c) & c^2\end{pmatrix}.
    \]
    Приравниваем к нулевой матрице:
    \[
    \begin{cases}
        a^2 = 0,\\
        c^2 = 0,\\
        b\,(a+c) = 0.
    \end{cases}
    \]
    В этом уравнении над полем \(\Bbb R\) из \(a^2=0\) и \(c^2=0\) сразу следует  
    \[
    a=0,\quad c=0.
    \]
    Тогда третье уравнение \(b\,(a+c)=b\cdot0=0\) выполняется при любом \(b\).

    Значит
    \[
    A^2=0\quad\Longleftrightarrow\quad a=0,\;c=0,
    \]
    Так $A^n = A^2\cdot A^{n-2}$ $n \geq 2$, то все нильпотентные элементы имеют вид:
    \[
    A=\begin{pmatrix}0&0\\b&0\end{pmatrix},\quad b\in\Bbb R.
    \]

    \item[\textbf{№2}]Допустим, что  
    \[
      I=(x-2,y)=(g)
    \]
    для некоторого \(g \in \QQ[x,y]\), \(g\neq0\).

    Тогда и \(x-2\in I\), и \(y\in I\) должны делиться на \(g\). То есть  
    \[
        g\mid(x-2)
        \quad\text{и}\quad
        g\mid y
        \quad\Longrightarrow\quad
        g\;\bigm|\;\gcd(x-2,\;y).
    \]
    Так как:
    \[
        \gcd(x-2,\;y)=1
    \]
    то любой их общий делитель \(g\) — обязательно обратимая константа из \(\QQ^\times\).
    
    Если \(g\) — единица, то  
    \[
        I=(g)=\QQ[x,y],
    \]
    то есть идеал совпадёт со всем кольцом. Но это невозможно, потому что, например,  \(1 \notin (x-2,y)\)

    Следовательно, этот идеал не главный\\

    \item[\textbf{№3}]Определим гомоморфизм колец \(\phi: \mathbb{C}[x] \to \mathbb{C} \oplus \mathbb{C}\) формулой:
    \[
    \phi(f(x)) = \left(f(0),\, f(-2)\right).
    \]
    Проверим корректность:
    \begin{itemize}
        \item \textit{Сложение:} 
        \[
        \phi(f + g) = \left((f + g)(0),\, (f + g)(-2)\right) = \left(f(0) + g(0),\, f(-2) + g(-2)\right) = \phi(f) + \phi(g)
        \]
        \item \textit{Умножение:} 
        \[
        \phi(f \cdot g) = \left((f \cdot g)(0),\, (f \cdot g)(-2)\right) = \left(f(0)g(0),\, f(-2)g(-2)\right) = \phi(f) \cdot \phi(g)
        \]
        \item \textit{Единица:} 
        \[
        \phi(1) = (1, 1)
        \]
    \end{itemize}

    Ядро гомоморфизма: \\
    \(\ker\phi\) состоит из многочленов \(f(x)\), для которых:
    \[
    f(0) = 0 \quad \text{и} \quad f(-2) = 0
    \]
    Так как \(x\) и \(x + 2\) взаимно просты, их произведение \(x(x + 2) = x^2 + 2x\) делит \(f(x)\). Следовательно:
    \[
    \ker\phi = (x^2 + 2x)
    \]

    Для любых \((a, b) \in \mathbb{C} \oplus \mathbb{C}\) построим многочлен \(f(x)\), такой что:
    \[
    f(0) = a, \quad f(-2) = b
    \]
    Например, подходит линейный многочлен:
    \[
    f(x) = \frac{a(x + 2) - b x}{2}
    \]
    Проверка:
    \[
    f(0) = \frac{a \cdot 2}{2} = a, \quad f(-2) = \frac{a \cdot 0 - b \cdot (-2)}{2} = b
    \]
    Значит, \(\phi\) сюръективен. Следовательно, $\Im \phi = (\mathbb{C}, \mathbb{C})$

    По теореме о гомоморфизме для колец:
    \[
    \mathbb{C}[x]/\ker\phi \cong \operatorname{Im}\phi
    \]
    Подставляя \(\ker\phi = (x^2 + 2x)\) и \(\operatorname{Im}\phi = \mathbb{C} \oplus \mathbb{C}\), получаем:
    \[
    \mathbb{C}[x]/(x^2 + 2x) \simeq \mathbb{C} \oplus \mathbb{C}
    \]

    \item[\textbf{№4}]\begin{enumerate}
        \item[(\(\Rightarrow\))] Если \( R/I \) --- поле, то \( I \neq R \) 
        и нет собственных идеалов \( J \), содержащих \( I \).
        \begin{itemize}
            \item Предположим, \( R/I \) --- поле. 
            Тогда \( R/I \) нетривиально (т.к. $0+I \neq 1 + I$), поэтому \( I \neq R \).
            \item Допустим, существует идеал \( J \triangleleft R \),
            такой что \( I \subsetneq J \subsetneq R \) ($J \subsetneq R$ т.к. в поле $0+J \neq 1 + J$). 
            Рассмотрим факторкольцо \( J/I \). Оно является идеалом в \( R/I \):

            (Для любого \( r + I \in R/I \) и \( j + I \in J/I \) произведение
             \( (r + I)(j + I) = rj + I \) принадлежит \( J/I \), так как \( J \) 
            --- идеал в \( R \) и \( rj \in J \).)
            
            Но поле не имеет собственных нетривиальных идеалов, кроме \(\{0\}\) и самого себя. 
            Следовательно, \( J/I = \{0\} \) (т.к. если \( J/I = R/I \implies J = R \) --- противоречит \( J \subsetneq R \)), откуда \( J = I \). 
            Это противоречит условию \( I \subsetneq J \).
            Значит, таких идеалов \( J \) не существует.
        \end{itemize}

        \item[(\(\Leftarrow\))] :(
    \end{enumerate}
\end{enumerate}
\end{document}