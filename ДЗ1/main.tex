\documentclass[a4paper]{article}
\usepackage{setspace}
\usepackage[T2A]{fontenc} %
\usepackage[utf8]{inputenc} % подключение русского языка
\usepackage[russian]{babel} %
\usepackage[12pt]{extsizes}
\usepackage{mathtools}
\usepackage{graphicx}
\usepackage{fancyhdr}
\usepackage{amssymb}
\usepackage{amsmath, amsfonts, amssymb, amsthm, mathtools}
\usepackage{tikz}

\usetikzlibrary{positioning}
\setstretch{1.3}

\newcommand{\mat}[1]{\begin{pmatrix} #1 \end{pmatrix}}
\renewcommand{\det}[1]{\begin{vmatrix} #1 \end{vmatrix}}
\renewcommand{\f}[2]{\frac{#1}{#2}}
\newcommand{\dspace}{\space\space}
\newcommand{\s}[2]{\sum\limits_{#1}^{#2}}
\newcommand{\mul}[2]{\prod_{#1}^{#2}}
\newcommand{\sq}[1]{\left[ {#1} \right]}
\newcommand{\gath}[1]{\left[ \begin{array}{@{}l@{}} #1 \end{array} \right.}
\newcommand{\case}[1]{\begin{cases} #1 \end{cases}}
\newcommand{\ts}{\text{\space}}
\newcommand{\lm}[1]{\underset{#1}{\lim}}
\newcommand{\suplm}[1]{\underset{#1}{\overline{\lim}}}
\newcommand{\inflm}[1]{\underset{#1}{\underline{\lim}}}

\renewcommand{\phi}{\varphi}
\newcommand{\lr}{\Leftrightarrow}
\renewcommand{\r}{\Rightarrow}
\newcommand{\rr}{\rightarrow}
\renewcommand{\geq}{\geqslant}
\renewcommand{\leq}{\leqslant}
\newcommand{\RR}{\mathbb{R}}
\newcommand{\CC}{\mathbb{C}}
\newcommand{\QQ}{\mathbb{Q}}
\newcommand{\ZZ}{\mathbb{Z}}
\newcommand{\VV}{\mathbb{V}}
\newcommand{\NN}{\mathbb{N}}
\newcommand{\OO}{\underline{O}}
\newcommand{\oo}{\overline{o}}
\newcommand{\divides}{\;|\;}
\newcommand{\leg}[2]{\left(\f{#1}{#2}\right)}

\DeclarePairedDelimiter\abs{\lvert}{\rvert} %
\makeatletter                               % \abs{}
\let\oldabs\abs                             %
\def\abs{\@ifstar{\oldabs}{\oldabs*}}       %

\begin{document}

\section*{Домашнее задание на 10.04 (Алгебра)}
 {\large Емельянов Владимир, ПМИ гр №247}\\\\
\begin{enumerate}
    \item[\textbf{№1}]Для любых \( m, n \in \mathbb{R} \setminus \{1\} \) проверим, что \( m \circ n \neq 1 \):
    \[
    m \circ n = 3mn - 3m - 3n + 4 = 1 \implies 3mn - 3m - 3n + 3 = 0 \implies (m - 1)(n - 1) = 0
    \]
    Но \( m, n \neq 1 \), следовательно, \( m \circ n \neq 1 \). Следовательно, это бинарная операция.
    
    Докажем, что это группа:
    \begin{enumerate}
        \item[1)]\textbf{Ассоциативность}:  
        
        Проверим, что \((a \circ b) \circ c = a \circ (b \circ c)\):
        \[
        (a \circ b) \circ c = (3ab - 3a - 3b + 4) \circ c = 3(3ab - 3a - 3b + 4)c - 3(3ab - 3a - 3b + 4) - 3c + 4
        \]
        Раскрыв скобки и приведя подобные, получим симметричное выражение относительно \( a, b, c \), что доказывает ассоциативность.
        

        \item[2)]\textbf{Нейтральный элемент:}
        
        Найдем \( e \), такой что \( m \circ e = m \):
        \[
        3me - 3m - 3e + 4 = m \implies e(3m - 3) = 4m - 4 \implies e = \frac{4}{3}
        \]
        Проверим, что \( \frac{4}{3} \) — нейтральный элемент:
        \[
        m \circ \frac{4}{3} = 3m \cdot \frac{4}{3} - 3m - 3 \cdot \frac{4}{3} + 4 = 4m - 3m - 4 + 4 = m
        \]
        \item[3)]\textbf{Обратный элемент:}
        
        Для \( m \) найдем \( m^{-1} \), такой что \( m \circ m^{-1} = \frac{4}{3} \):
        \[
        3m m^{-1} - 3m - 3m^{-1} + 4 = \frac{4}{3} \implies m^{-1}(3m - 3) = \frac{4}{3} + 3m - 4
        \]
        \[
        m^{-1} = \frac{9m - 8}{9(m - 1)}
        \]
        Так как \( m \neq 1 \), обратный элемент существует.
        
    \end{enumerate}

    \item[\textbf{№2}]Определим все значения \( a \geq 1 \), при которых \( H_a = \{x \in \mathbb{R} \mid x > a\} \) является подгруппой в \( G \).
    
    \textbf{1. Нейтральный элемент:}
    \[
    \frac{4}{3} > a \implies a < \frac{4}{3}
    \]
    Но по условию \( a \geq 1 \), поэтому \( a \in \left[1, \frac{4}{3}\right) \).
    
    \textbf{2. Замкнутость:}
    Для \( x, y > a \) проверим \( x \circ y > a \):
    \[
    x \circ y = 3xy - 3x - 3y + 4 > a
    \]
    Это неравенство выполняется для \( a \geq \frac{4}{3} \), так как при \( x, y \to a^+ \) минимальное значение \( x \circ y \) стремится к \( 3a^2 - 6a + 4 \geq a \).
    
    \textbf{3. Обратный элемент:}
    Для \( x > a \) проверим \( x^{-1} > a \):
    \[
    x^{-1} = \frac{9x - 8}{9(x - 1)} > a
    \]
    Решая неравенство, получаем \( x > \frac{9a - 8}{9a - 9} \). Для \( a \geq \frac{4}{3} \) это выполняется.
    
    \textbf{Ответ:} \( a \in \left[\frac{4}{3}, +\infty\right) \).

    \item[\textbf{№3}]Для группы \( (\mathbb{Z}_{17}\backslash\{0\}, \times) \) найдем порядки и обратные элементы:

    \begin{center}
    \begin{tabular}{|c|c|c|}
    \hline
    Элемент \( x \) & Порядок \( \text{ord}(x) \) & Обратный \( x^{-1} \) \\
    \hline
    1 & 1 & 1 \\
    2 & 8 & 9 \\
    3 & 16 & 6 \\
    4 & 4 & 13 \\
    5 & 16 & 7 \\
    6 & 16 & 3 \\
    7 & 16 & 5 \\
    8 & 8 & 15 \\
    9 & 8 & 2 \\
    10 & 16 & 12 \\
    11 & 16 & 14 \\
    12 & 16 & 10 \\
    13 & 4 & 4 \\
    14 & 16 & 11 \\
    15 & 8 & 8 \\
    16 & 2 & 16 \\
    \hline
    \end{tabular}
    \end{center}
    \;\\

    \item[\textbf{№4}]Докажем, что всякая подгруппа циклической группы является циклической.

    \textbf{Доказательство:}
    Пусть \( G = \langle g \rangle \) — циклическая группа, \( H \subseteq G \).
    
    \begin{enumerate}
        \item Если \( H = \{e\} \), то \( H = \langle e \rangle \) — циклическая.
        \item Иначе, рассмотрим множество \( S = \{k \in \mathbb{N} \mid g^k \in H\} \). Оно непусто, так как \( H \) содержит хотя бы один элемент \( g^n \neq e \).
        \item Пусть \( d \) — наименьший натуральный элемент \( S \). Покажем, что \( H = \langle g^d \rangle \):
        \begin{itemize}
            \item \( \langle g^d \rangle \subseteq H \), так как \( g^d \in H \).
            \item Для любого \( h \in H \) существует \( k \), такое что \( h = g^k \). Разделим \( k \) на \( d \) с остатком: \( k = qd + r \), \( 0 \leq r < d \). Тогда \( g^r = g^{k - qd} = h (g^d)^{-q} \in H \). Из минимальности \( d \) следует \( r = 0 \), значит, \( h = (g^d)^q \in \langle g^d \rangle \).
        \end{itemize}
    \end{enumerate}
    
    Таким образом, \( H \) — циклическая группа.
    
\end{enumerate}
\end{document}