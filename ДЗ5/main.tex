\documentclass[a4paper]{article}
\usepackage{setspace}
\usepackage[utf8]{inputenc}
\usepackage[russian]{babel}
\usepackage[12pt]{extsizes}
\usepackage{mathtools}
\usepackage{graphicx}
\usepackage{fancyhdr}
\usepackage{amssymb}
\usepackage{amsmath, amsfonts, amssymb, amsthm, mathtools}
\usepackage{tikz}

\usetikzlibrary{positioning}
\setstretch{1.3}

\newcommand{\mat}[1]{\begin{pmatrix} #1 \end{pmatrix}}
\newcommand{\vmat}[1]{\begin{vmatrix} #1 \end{vmatrix}}
\renewcommand{\f}[2]{\frac{#1}{#2}}
\newcommand{\dspace}{\space\space}
\newcommand{\s}[2]{\sum\limits_{#1}^{#2}}
\newcommand{\mul}[2]{\prod_{#1}^{#2}}
\newcommand{\sq}[1]{\left[ {#1} \right]}
\newcommand{\gath}[1]{\left[ \begin{array}{@{}l@{}} #1 \end{array} \right.}
\newcommand{\case}[1]{\begin{cases} #1 \end{cases}}
\newcommand{\ts}{\text{\space}}
\newcommand{\lm}[1]{\underset{#1}{\lim}}
\newcommand{\suplm}[1]{\underset{#1}{\overline{\lim}}}
\newcommand{\inflm}[1]{\underset{#1}{\underline{\lim}}}
\newcommand{\Ker}[1]{\operatorname{Ker}}

\renewcommand{\phi}{\varphi}
\newcommand{\lr}{\Leftrightarrow}
\renewcommand{\l}{\left(}
\renewcommand{\r}{\right)}
\newcommand{\rr}{\rightarrow}
\renewcommand{\geq}{\geqslant}
\renewcommand{\leq}{\leqslant}
\newcommand{\RR}{\mathbb{R}}
\newcommand{\CC}{\mathbb{C}}
\newcommand{\QQ}{\mathbb{Q}}
\newcommand{\ZZ}{\mathbb{Z}}
\newcommand{\VV}{\mathbb{V}}
\newcommand{\NN}{\mathbb{N}}
\newcommand{\OO}{\underline{O}}
\newcommand{\oo}{\overline{o}}
\renewcommand{\Ker}{\operatorname{Ker}}
\renewcommand{\Im}{\operatorname{Im}}
\newcommand{\vol}{\text{vol}}
\newcommand{\Vol}{\text{Vol}}

\DeclarePairedDelimiter\abs{\lvert}{\rvert} %
\makeatletter                               % \abs{}
\let\oldabs\abs                             %
\def\abs{\@ifstar{\oldabs}{\oldabs*}}       %

\begin{document}

\section*{Домашнее задание на 15.05 (Алгебра)}
{\large Емельянов Владимир, ПМИ гр №247}\\\\
\begin{enumerate}
  \item[\textbf{№1}]\begin{enumerate}
    \item[(a)]Пусть
    \[
    f(x)=x^5+x^4+x^2-4x-2,\qquad
    g(x)=2x^4 - x^3 -2x^2 -2x -12.
    \]
    Применим алгоритм Евклида.
    \begin{align*}
    f(x) &= \Bigl(\tfrac12 x + \tfrac32\Bigr)\,g(x) + r_1(x), &\ deg(r_1)=3,\\
    r_1(x) &= f(x) - (\tfrac12 x + \tfrac32)g(x) = \dots = \frac{5}{2}x^3 + \frac{7}{2}x^2 + x + 16,\\
    g(x) &= q_2(x)\,r_1(x) + r_2(x), &\ deg(r_2)=2,\\
    r_2(x) &= g(x) - q_2(x)r_1(x) = \dots = x^2 + 2,\\
    r_1(x) &= q_3(x)\,r_2(x) + r_3(x), &\ deg(r_3)<2,\\
    r_3(x) &= r_1(x) - q_3(x)r_2(x) = 0.
    \end{align*}
    Вычисления дают
    \[
    \gcd\bigl(f,g\bigr)=x^2+2,
    \]

    Теперь восстанавливаем коэффициенты.
    \begin{align*}
    r_2(x) &= g(x) - q_2(x)r_1(x)\\
    r_1(x) &= f(x) - \Bigl(\tfrac12 x + \tfrac32\Bigr)g(x)
    \end{align*}
    Подставляя, получаем
    \[
    x^2+2 = s(x)f(x) + t(x)g(x),
    \]
    где
    \[
    s(x)=\frac{5-2x}{7},\quad t(x)=\frac{x^2 - x -2}{7}.
    \]

    такие, что
    \[
    s(x)\,f(x) + t(x)\,g(x) \;=\; x^2+2.
    \]

    \item[(b)]$K=\mathbb{Z}_7$
    
    Рассмотрим
    \[
    f(x)=x^5+x^3+3x^2+5x+3,\quad g(x)=3x^4+3x^3+6x+1
    \]
    применим алгоритм Евклида в $\mathbb{Z}_7[x]$:
    \begin{align*}
    f(x)&=5x\,g(x) + r_1(x), & r_1(x)=f-5xg=6x^4+x^3+x^2+3,\\
    g(x)&=4\,r_1(x) + r_2(x), & r_2(x)=g-4r_1=6x^3+3x^2+6x+3,\\
    r_1(x)&=x\,r_2(x) + r_3(x), & r_3(x)=r_1-xr_2= x^2+4x+5,\\
    r_2(x)&=(6x+1)\,r_3(x) + r_4(x), & r_4(x)=r_2-(6x+1)r_3= x-3,\\
    r_3(x)&=3x+6\,r_4(x) +0.
    \end{align*}
  \end{enumerate}
\end{enumerate}
\end{document}