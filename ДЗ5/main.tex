\documentclass[a4paper]{article}
\usepackage{setspace}
\usepackage[utf8]{inputenc}
\usepackage[russian]{babel}
\usepackage[12pt]{extsizes}
\usepackage{mathtools}
\usepackage{graphicx}
\usepackage{fancyhdr}
\usepackage{amssymb}
\usepackage{amsmath, amsfonts, amssymb, amsthm, mathtools}
\usepackage{tikz}

\usetikzlibrary{positioning}
\setstretch{1.3}

\newcommand{\mat}[1]{\begin{pmatrix} #1 \end{pmatrix}}
\newcommand{\vmat}[1]{\begin{vmatrix} #1 \end{vmatrix}}
\renewcommand{\f}[2]{\frac{#1}{#2}}
\newcommand{\dspace}{\space\space}
\newcommand{\s}[2]{\sum\limits_{#1}^{#2}}
\newcommand{\mul}[2]{\prod_{#1}^{#2}}
\newcommand{\sq}[1]{\left[ {#1} \right]}
\newcommand{\gath}[1]{\left[ \begin{array}{@{}l@{}} #1 \end{array} \right.}
\newcommand{\case}[1]{\begin{cases} #1 \end{cases}}
\newcommand{\ts}{\text{\space}}
\newcommand{\lm}[1]{\underset{#1}{\lim}}
\newcommand{\suplm}[1]{\underset{#1}{\overline{\lim}}}
\newcommand{\inflm}[1]{\underset{#1}{\underline{\lim}}}
\newcommand{\Ker}[1]{\operatorname{Ker}}

\renewcommand{\phi}{\varphi}
\newcommand{\lr}{\Leftrightarrow}
\renewcommand{\l}{\left(}
\renewcommand{\r}{\right)}
\newcommand{\rr}{\rightarrow}
\renewcommand{\geq}{\geqslant}
\renewcommand{\leq}{\leqslant}
\newcommand{\RR}{\mathbb{R}}
\newcommand{\CC}{\mathbb{C}}
\newcommand{\QQ}{\mathbb{Q}}
\newcommand{\ZZ}{\mathbb{Z}}
\newcommand{\VV}{\mathbb{V}}
\newcommand{\NN}{\mathbb{N}}
\newcommand{\OO}{\underline{O}}
\newcommand{\oo}{\overline{o}}
\renewcommand{\Ker}{\operatorname{Ker}}
\renewcommand{\Im}{\operatorname{Im}}
\newcommand{\vol}{\text{vol}}
\newcommand{\Vol}{\text{Vol}}

\DeclarePairedDelimiter\abs{\lvert}{\rvert} %
\makeatletter                               % \abs{}
\let\oldabs\abs                             %
\def\abs{\@ifstar{\oldabs}{\oldabs*}}       %

\begin{document}

\section*{Домашнее задание на 15.05 (Алгебра)}
{\large Емельянов Владимир, ПМИ гр №247}\\\\
\begin{enumerate}
  \item[\textbf{№1}]\begin{enumerate}
    \item[(a)]Пусть
    \[
    f(x)=x^5+x^4+x^2-4x-2,\qquad
    g(x)=2x^4 - x^3 -2x^2 -2x -12.
    \]
    Применим алгоритм Евклида.
    \begin{align*}
    f(x) &= \Bigl(\tfrac12 x + \tfrac32\Bigr)\,g(x) + r_1(x), &\ deg(r_1)=3,\\
    r_1(x) &= f(x) - (\tfrac12 x + \tfrac32)g(x) = \dots = \frac{5}{2}x^3 + \frac{7}{2}x^2 + x + 16,\\
    g(x) &= q_2(x)\,r_1(x) + r_2(x), &\ deg(r_2)=2,\\
    r_2(x) &= g(x) - q_2(x)r_1(x) = \dots = x^2 + 2,\\
    r_1(x) &= q_3(x)\,r_2(x) + r_3(x), &\ deg(r_3)<2,\\
    r_3(x) &= r_1(x) - q_3(x)r_2(x) = 0.
    \end{align*}
    Вычисления дают
    \[
    \gcd\bigl(f,g\bigr)=x^2+2,
    \]

    Теперь восстанавливаем коэффициенты.
    \begin{align*}
    r_2(x) &= g(x) - q_2(x)r_1(x)\\
    r_1(x) &= f(x) - \Bigl(\tfrac12 x + \tfrac32\Bigr)g(x)
    \end{align*}
    Подставляя, получаем
    \[
    x^2+2 = s(x)f(x) + t(x)g(x),
    \]
    где
    \[
    s(x)=\frac{5-2x}{7},\quad t(x)=\frac{x^2 - x -2}{7}
    \]

    такие, что
    \[
    s(x)\,f(x) + t(x)\,g(x) \;=\; x^2+2
    \]

    \item[(b)]$K=\mathbb{Z}_7$
    
    Рассмотрим
    \[
    f(x)=x^5+x^3+3x^2+5x+3,\quad g(x)=3x^4+3x^3+6x+1
    \]
    применим алгоритм Евклида в $\mathbb{Z}_7[x]$:
    \begin{align*}
    f(x)&=5x\,g(x) + r_1(x), & r_1(x)=f-5xg=6x^4+x^3+x^2+3,\\
    g(x)&=4\,r_1(x) + r_2(x), & r_2(x)=g-4r_1=6x^3+3x^2+6x+3,\\
    r_1(x)&=x\,r_2(x) + r_3(x), & r_3(x)=r_1-xr_2= x^2+4x+5,\\
    r_2(x)&=(6x+1)\,r_3(x) + r_4(x), & r_4(x)=r_2-(6x+1)r_3= x-3,\\
    r_3(x)&=3x+6\,r_4(x) +0.
    \end{align*}
    Отсюда 
    $$\gcd(f,g)=r_4(x)=x-3$$
    Восстановим коэффициенты:
    \begin{align*}
    r_4(x)&=r_2(x)-(6x+1)r_3(x),\\
    r_3(x)&=r_1(x)-xr_2(x),\\
    r_2(x)&=g(x)-4r_1(x),\\
    r_1(x)&=f(x)-5xg(x).
    \end{align*}
    В итоге
    \[
    x-3=u(x)f(x)+v(x)g(x),\quad u(x)=3x^2+4x+1,\ v(x)=6x^3+2x^2+6x+1
    \]
  \end{enumerate}

  \item[\textbf{№2}]
  \begin{enumerate}
    \item[(a)] \(f(x)=x^5+3x^3-4x^2-12\)
    
    Заметим, что
    \[
      x^5+3x^3-4x^2-12
      \;=\;
      (x^3-4)(x^2+3),
    \]

    получаем следующие разложения:
    \begin{itemize}
      \item При \(K=\mathbb{R}\) многочлен \(x^3-4\) даёт единственный действительный корень \(\sqrt[3]{4}\), а \(x^2+3\) неприводим. 
      \[
        f(x) \;=\; (x^3-4)(x^2+3)
      \]
      \item При \(K=\mathbb{C}\) раскладываем оба множителя на линейные:
      \[
        x^3-4 \;=\;\prod_{k=0}^{2}\Bigl(x-\sqrt[3]{4}\,e^{2\pi i k/3}\Bigr),
        \qquad
        x^2+3=(x-i\sqrt3)(x+i\sqrt3)
      \]
      Значит,
      \[
        f(x)
        \;=\;
        \bigl(x-\sqrt[3]{4}\bigr)\,
        \bigl(x-\sqrt[3]{4}\,e^{2\pi i/3}\bigr)\,
        \bigl(x-\sqrt[3]{4}\,e^{4\pi i/3}\bigr)\,
        (x-i\sqrt3)(x+i\sqrt3)
      \]\\
    \end{itemize}

    \item[(b)]\(K=\mathbb{Z}_5\), \(f(x)=x^5+x^4+3x^2+x+3\in\mathbb{Z}_5[x]\)
    Проверкой всех возможных корней и факторизацией по модулю 5 получаем
    \[
      f(x) \;=\;
      (x-2)^2\,(x+1)\,(x^2 - x + 2)
      \quad(\bmod 5),
    \]
    где квадратный трёхчлен \(x^2 - x + 2\) неприводим над \(\mathbb{Z}_5\)\\
  \end{enumerate}

  \item[\textbf{№3}]Пусть \( \ZZ_3 = \{0,1,2\}\) --- поле вычетов по модулю 3.
   Перечислим все неприводимые многочлены со старшим коэфицентом 1, степеней \(1,2,3\) и укажем число таких степени 4.
  \begin{enumerate}
    \item \(\deg=1\). линейные многочлены:
    \[
      x,\quad x+1,\quad x+2.
    \]
    \item \(\deg=2\). квадратичные многочлены без корней в \(\ZZ_3\):
    \[
      x^2 + 1,\quad x^2 + x + 2,\quad x^2 + 2x + 2.
    \]
    
    \item \(\deg=3\). кубические многочлены без корней в \(\ZZ_3\):
    \[
      \begin{gathered}
        x^3 + 2x + 1,\quad x^3 + 2x + 2,\\
        x^3 + x^2 + 2,\quad x^3 + x^2 + x + 2,\quad x^3 + x^2 + 2x + 1,\\
        x^3 + 2x^2 + 1,\quad x^3 + 2x^2 + x + 1,\quad x^3 + 2x^2 + 2x + 2.
      \end{gathered}
    \]
    
    \item \(\deg=4\). 
    
    Рассмотрим все многочлены вида
    \[
      f(x)=x^4 + a x^3 + b x^2 + c x + d,\quad a,b,c,d\in\ZZ_3
    \]
    Всего таких многочленов \(\lvert\ZZ_3\rvert^4 = 3^4 = 81\).

    \begin{itemize}
      \item \textbf{Многочлены, имеющие линейный множитель (корень) в \(\ZZ_3\).}\\
        Для каждого \(\alpha\in\{0,1,2\}\) число многочленов с корнем \(\alpha\) равно \(3^3\).  
        По принципу включений–исключений получаем
        \[
          N_{\rm с\ корнем}
          \;=\;
          3\cdot 3^3
          \;-\;
          \binom{3}{2}\,3^2
          \;+\;
          \binom{3}{3}\,3
          \;=\;
          81 - 27 + 3
          \;=\;
          57
        \]
    
      \item \textbf{Многочлены, раскладывающиеся в произведение двух неприводимых квадратичных (но не имеющие линейных множителей).}\\
        Из предыдущего пункта известно, что неприводимых квадратичных над \(\ZZ_3\) ровно 3.  
        Число их попарных произведений (с повторениями) равно
        \[
          \binom{3}{2} + 3 = 3 + 3 = 6
        \]
    \end{itemize}
      
    Следовательно, число оставшихся (неприводимых) многочленов степени 4:
    \[
      81 \;-\; 57 \;-\; 6 \;=\; 18
    \]\\
  \end{enumerate}
  
  \item[\textbf{№4}]
  
  Пусть \(f(x),g(x)\in \QQ [x]\) --- два неприводимых многочлена, и пусть они имеют общий корень \(r \in \CC\). Тогда в кольце \(\QQ[x]\) их наибольший общий делитель \(\gcd(f,g)\) --- не константа, а значит его степень \(\ge 1\).

  Но неприводимость каждого из них означает, что единственные делители (с ненулевой степенью) --- это они сами (с точностью до константы). Поэтому
  \[
  \gcd(f,g) \sim  f
  \quad\text{и}\quad
  \gcd(f,g) \sim  g.
  \]
  Отсюда \(f\mid g\) и \(g\mid f\), что возможно только если
  \[
  g(x)=c\,f(x)
  \quad\text{для некоторого }c\in \QQ^\times.
  \]
  Иными словами, \(f\) и \(g\) отличаются лишь константой, то есть пропорциональны.
  
\end{enumerate}
\end{document}